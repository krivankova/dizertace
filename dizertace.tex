 \documentclass[a4paper,12pt]{report}
  \pagestyle{plain}
  \usepackage{latexsym}      %symboly
  \usepackage{graphicx}
  \usepackage{amsmath}
  \usepackage{amsfonts}
  \usepackage{amsthm}
  \usepackage{epstopdf} %pokud pouzivam .eps obrazky, ale chci prekladat pdflatexem 
  \usepackage[czech]{babel}
  \usepackage[utf8]{inputenc}
  \usepackage{comment}

\usepackage{url}  %kvuli vkladani URL adres do bibtexu ale asi jen ve stylu CZECHISO
\DeclareUrlCommand\url{\def\UrlLeft{<}\def\UrlRight{>} \urlstyle{tt}}
  
%\renewcommand{\figurename}{Fig.}
  \usepackage{phdthesis}
  \usepackage{psfrag}
%  \newtheorem{definice}{Definition}
  \usepackage{hyperref} % klikatelne odkazy v textu
  \hypersetup{
      colorlinks   = true, %Colours links instead of ugly boxes
      urlcolor     = blue, %Colour for external hyperlinks
      linkcolor    = red, %Colour of internal links
      citecolor   = red %Colour of citations
    }
  \date{\today}
  \usepackage{minibox}
  \usepackage{pdfpages}
%
%
\vfuzz10pt % Don't report over-full v-boxes if over-edge is small
\hfuzz10pt
%
%
%\newtheorem{theorem}{Theorem}[chapter]
%\newtheorem{definition}{Definition}[chapter]
%\newtheorem{example}{Example}[chapter]
\DeclareMathOperator{\limean}{l.i.m.}   %limit in the mean se stane operatorem
\DeclareMathOperator{\argmax}{argmax}   %stejne tak argmin a argmax
\DeclareMathOperator{\argmin}{argmin}
%
%

\newtheorem{veta}{Věta} 
\newtheorem{lemma}[veta]{Lemma}
\newtheorem{dusledek}[veta]{Důsledek}
\theoremstyle{definition} \newtheorem{definice}[veta]{Definice}
\newtheorem{priklad}{Příklad}
\theoremstyle{remark}
\newtheorem*{poznamka}{Poznámka}
\newtheorem*{reseni}{Řešení}  



\newcommand{\specialcell}[2][c]{%
\begin{tabular}[#1]{@{}l@{}}#2\end{tabular}}



\begin{document}
\titlepage
\vspace*{-2.4cm}
\begin{center}
\textsc{\Large{Masarykova Univerzita}} \\
\large{Přírodovědecká fakulta} \\
\large{Ústav matematiky a statistiky}
\end{center}
\vspace{7.11cm}
%
\begin{center}
\textsc{\Large{DISERTAČNÍ PRÁCE}}
\end{center}
%
\vspace{11.39cm}
%
\begin{center}
\begin{tabular}{l  r}
 \Large{Brno 2015} \qquad\qquad\qquad\qquad \qquad\qquad & \Large{Lenka Křivánková}
\end{tabular}
\end{center}

\titlepage
\begin{center}
\textsc{\Large{Masarykova Univerzita}} \\
\textsc{\Large{Přírodovědecká fakulta}} \\
\textsc{\large{Ústav matematiky a statistiky}}
\end{center}\quad \\
%
\begin{center}
\includegraphics[width=5cm]{IMG/sci-logo.pdf}
\end{center}\quad \\
%
\begin{center}
\textbf{\Large{Stochastické metody analýzy ekonomických dat}} \\ \quad \\
\textsc{\large{Disertační práce}}
\end{center} \quad \\\\\\\\\\\\\\\\\\\\\\\\\\ \\
\begin{center}
\large{
\begin{tabular}{l  r}
\Large{Brno 2015} \qquad\qquad\qquad\qquad \qquad\qquad & \Large{Lenka Křivánková}
\end{tabular}
}
\end{center} \quad
%
%\begin{center}
%\large{Brno 2009}
%\end{center}
%\begin{center}
%\verb=   (\(\=\\
%\verb=  (._. )=\\
%\verb=(')(')_)o=
%\end{center}

\normalsize



\chapter*{Bibliografický záznam}\addcontentsline{toc}{chapter}{Bibliografický záznam}
\setcounter{page}{1}
\pagenumbering{roman}
%\thispagestyle{empty}

\begin{tabular}{p{3.9cm}  p{9.1cm}}
\textbf{Autor:} & Mgr. Lenka Křivánková \\
\textbf{Název:} & Stochastické metody analýzy ekonomických dat \\
\textbf{Školitel:} & doc. RNDr. Martin Kolář, Ph.D.  \\
\textbf{Studijní program:} & Matematika \\
\textbf{Obor:} & Pravděpodobnost, statistika a matematické modelování \\
\textbf{Rok obhajoby:} & 2015 \\
\textbf{Klíčová slova:} & Stochastické procesy, \\
\end{tabular}
\\\\\\\\\\

\begin{flushright} {{\Huge Bibliographic entry}} \vspace{38pt} \end{flushright}
%\addcontentsline{toc}{chapter}{Bibliographic entry}

\hspace{-0.7cm}
\begin{tabular}{p{3.9cm}  p{9.1cm}}
\textbf{Author:} & Mgr. Lenka Křivánková \\
\textbf{Title:} & Stochastic methods in analysis of economic data \\
\textbf{Supervisor:} & doc. RNDr. Martin Kolář, Ph.D.  \\
\textbf{Study programme:} & Mathematics \\
\textbf{Study field:} & Probability, statistics and mathematical modeling \\
\textbf{Year of defence:} & 2015 \\
\textbf{Keywords:} & Stochastic processes, \\
\end{tabular}
\newpage
\pagestyle{empty}
\null
\vfill
\begin{center}
\copyright \quad Lenka Křivánková, Masarykova Univerzita, 2015
\end{center}

\chapter*{Poděkování}\addcontentsline{toc}{chapter}{Poděkování}
%\thispagestyle{empty}
Na tomto místě
\\

Brno, Listopad 2015
\\

%Jakub \u Cupera

\chapter*{Abstrakt}\addcontentsline{toc}{chapter}{Abstrakt}
%\thispagestyle{empty}
V první části práce jsou uvedeny základní matematické pojmy použité v dalších částech textu. 
\\\\\\\\\\

\begin{flushright} {{\Huge Abstract}} \vspace{38pt} \end{flushright}
%\addcontentsline{toc}{chapter}{Abstract}
In the first part of the work, some basic mathematical methods employed in this thesis are recalled. 

\tableofcontents \addcontentsline{toc}{chapter}{Obsah}

\chapter*{Seznam použitého značení}\addcontentsline{toc}{chapter}{Seznam použitého značení}
%\hspace{-0.7cm}
%\begin{tabular}{r  l}
%$\mathsf{E}X$ & Expected value of random variable $X$ \\
%$\mathsf{var}X$ & Dispersion, resp. variance of random variable X \\
%\end{tabular}

\normalsize
\textbf{Pravděpodobnost}\\\\
%  \begin{table}[h]
   \begin{tabular}{p{4cm} p{9.3cm}}
   $\Omega$                                          &   základní prostor, množina všech elementárních jevů \\
   $\mathsf{Pr}(A)$                               &  pravděpodobnost jevu $A$ \\
   $(\Omega,\mathcal{A}, \mathsf{Pr})$                         &   pravděpodobnostní prostor \\
   \end{tabular}\\\\\\
%
%  \end{table}
%
%
\textbf{Náhodné veličiny}\\\\
%  \begin{table}[h]
   \begin{tabular}{p{4cm} p{9.3cm}}
   $X$                                              &   náhodná veličina \\
   $\mathsf{E}(X)$                          &   střední hodnota náhodné veličiny $X$ \\
   $\mathsf{D}(X)$                          &   rozptyl náhodné veličiny $X$ \\
   $\mathsf{C}(X_i,X_j)$                 &   kovariance náhodných veličin $X_i$ a $X_j$\\
%   $X(t)$; $X_t$                                     &    continuous-time random process \\
   $W(t); W_t$                                 &    standardní Wienerův proces \\
%   $f(x)$                                            &    probability density of the random variable $X$ \\
%   $f(x; \theta)$                                            &   probability density function of random variable $X$ with distribution dependent on scalar parameter $\theta$ \\
%   $f(x,t)$                                          &    probability density of random process $X(t)$ at fixed time $t$ \\
%   $f(x,t | x_0, t_0)$                               &    probability density of random process $X(t)$ at fixed time $t$ conditioned by its state $x_0$ at time $t_0$ \\
%   $X\sim \mathcal{L}$                               &    random variable $X$ having general probability distribution $\mathcal{L}$\\
   $\mathsf{N}(\mu, \sigma^{2})$                     &   Normální (Gaussovo) rozdělení pravděpodobnosti se střední hodnotou $\mu$ a rozptylem $\sigma^2$  \\
   \end{tabular}\\\\\\
%
%
   \textbf{Náhodné vektory}\\\\
%  \begin{table}[h]
   \begin{tabular}{p{4cm} p{9.3cm}}
  $\boldsymbol{X}=(X_1,\dots,X_n)^\mathrm{T}$                 &     $n$-rozměrný náhodný vektor \\
%   $\boldsymbol{\theta}=(\theta_1,\ldots,\theta_m)$  &    m-dimensional vector parameter \\
%   $f(\mathbf{x};\boldsymbol{\theta})$               &    probability density of the random vector $\boldsymbol{X}$ with vector parameter $\boldsymbol{\theta}$ \\
   $\mathsf{E}\boldsymbol{X}$                             &     střední hodnota náhodného vektoru $\boldsymbol{X}$\\
   $\mathsf{D}\boldsymbol{X}$                             &   rozptyl  náhodného vektoru $\boldsymbol{X}$\\
   \end{tabular}\\\\\\
%
%
%
   \textbf{Prostory a matice}\\\\
%  \begin{table}[h]
   \begin{tabular}{p{4cm} p{9.3cm}}
   $\mathbb{R}$                              &   jednorozměrný Eukleidovský prostor \\
   $\mathbb{R}^n$                              &     $n$-rozměrný Eukleidovský prostor \\
   $\boldsymbol{x}=(x_1,\ldots,x_n)^\mathrm{T}$             &    $n$-rozměrný reálný vektor \\
    $\boldsymbol{1}$                           &   vektor jedniček\\
   $\mathbf{I}_n$                              &    jednotková matice řádu $n$ \\
   $\mathbf{A}^T$                              &   transponovaná matice vzhledem k matici $\mathbf{A}$\\
   \end{tabular}
% \end{table}
  %
%
%


\chapter{Úvod}
\setcounter{page}{1}
\pagenumbering{arabic}

\begin{comment}
Klasická kniha o Stochastických diferenciálních rovnicích \cite{oksendal2003stochastic}\\
Numerické řešení stochastických diferenciálních rovnic a jejich soustav \cite{kloeden1997}\\
Vzorec pro váhy tržního portfolia čerpáme z \cite{fabozzi}\\
Základy teorie portfolia pokládá Markowitz v \cite{markowitz}\\
Tobin v \cite{tobin} rozšiřuje Markowitzův přístup o bezrizikové aktivum.
Capital Asset Pricing Model (CAPM), který nezávisle na sobě zkonstruovali Sharpe (1964) v \cite{sharpe1964}, Lintner (1965) v \cite{lintner1965} a Mossin (1966) v \cite{mossin1966} zkoumá chování trhu v případě, že se všichni investoři chovají podle Markowitzovy teorie.
První řešení spojitého modelu v teorii portfolia dává Merton v \cite{merton1971}\\
Na rozpor předpokladů v Mertonově článku \cite{merton1971} upozorňují Ohlson a Rosenberg v \cite{ohlson}\\
Na jejich článek reaguje Merton v \cite{merton1975} a jeho reakce vychází ješte před článkem Ohlsona a Rosenberga.\\

Konzistentní dynamický přístup k teorii portfolia dává Fernholz ve své knize Stochastická teorie portfolia \cite{fern} a v článku s Karatzasem \cite{kara}\\
\end{comment}

\chapter{Základní matematické pojmy}
%\chapter{Teoretická východiska}

\section{Wienerův proces}

\begin{definice}
Reálný stochastický proces $\{W(t):t\ge0\}$ na pravděpodobnostním prostoru $(\Omega,\mathcal{A},\mathsf{Pr})$ se nazývá \textit{Wienerův proces}, jestliže platí
\begin{enumerate}
\item[1.]$W(0)=0$, 
\item[2.](spojitost trajektorií Wienerova procesu) s pravděpodobností jedna je funkce $t\to W(t)$ spojitá v $t$,
\item[3.](nezávislost a stacionarita přírůstků) přírůstky procesu jsou nezávislé a stacionární, t.j.  pro každé $t\ge s\ge0$ má $X(t+s)-X(s)$ stejné rozdělení jako $X(t)-X(0)$,
\item[4.](normalita přírůstků) pro každé $t\ge s\ge0$ má přírůstek $W(t)-W(s)$ normální rozdělení $\mathsf{N}(0, t-s)$.
\end{enumerate}
\end{definice}

\subsection{Wienerův proces s driftem}
\begin{definice}
Stochastický proces $X(t)$ definovaný 
$$X(t)  = \mu t + \sigma W (t)$$
se nazývá \textit{Wienerův proces  s driftem} $\mu$ a volatilitou $\sigma$.
\end{definice}

\section{It\^oův proces}

\begin{definice}
Nechť $W(t,\omega)$ je Wienerův proces na pravděpodobnostním prostoru $(\Omega,\mathcal{A},\mathsf{Pr})$.
\textit{It\^oův proces} je stochastický proces tvaru
$$X(t,\omega)=\int_0^tU(s,\omega)\mathrm{d}s+\int_0^tV(s,\omega)\mathrm{d}W(s,\omega),$$
kde $U(t,\omega)$ a $V(t,\omega)$ jsou stochastické procesy patřící do $M$.

Často se používá diferenciální tvar, nazývaný \textit{stochastický diferenciál},
$$\mathrm{d}X(t,\omega)=U(t,\omega)\mathrm{d}t+V(t,\omega)\mathrm{d}W(t,\omega).$$
\end{definice}


\subsection{Geometrický Wienerův proces}
\begin{definice}
Stochastický proces $S(t)$ definovaný 
$$\frac{\mathrm{d} S(t)}{S(t)} = \mu\mathrm{d}t + \sigma\mathrm{d}W (t)$$
se nazývá \textit{geometrický Wienerův proces}.
\end{definice}

\section{It\^oovo lemma}
It\^oovo lemma je nástrojem pro práci s It\^oovým integrálem a stochastickými diferenciálními rovnicemi.
It\^oovo lemma říká, že It\^oovy procesy tvoří uzavřenou třídu vzhledem ke skládání s hladkými funkcemi.

\begin{veta}[It\^oovo lemma]
Nechť $X(t,\omega)$ je It\^oův proces se stochastickým diferenciálem
$$\mathrm{d}X=U\mathrm{d}t+V\mathrm{d}W.$$
Nechť $g(t,x):(0,\infty)\times\mathbb R\to\mathbb R$ je dvakrát spojitě diferencovatelná funkce.
Potom $Y(t)=g(t,X(t))$ je také It\^oův proces.
Jeho stochastický diferenciál má tvar
\begin{alignat*}{2}
\mathrm{d}Y=\frac{\partial g}{\partial t}\mathrm{d}t+\frac{\partial g}{\partial x}\,\mathrm{d}X+\frac12\frac{\partial^2 g}{\partial x^2}(\mathrm{d}X)^2=\left[\frac{\partial g}{\partial t}+\frac{\partial g}{\partial x}U+\frac12\frac{\partial^2 g}{\partial x^2}V^2\right]\mathrm{d}t+\frac{\partial g}{\partial x}V\mathrm{d}W.
\end{alignat*}
\end{veta}

\section{It\^oův integrál}
Důležitým nástrojem pro počítání stochastických diferenciálních rovnic je It\^oův integrál.
Budeme ho definovat obdobným způsobem jako Riemannův integrál.
Nejprve It\^oův integrál definujeme pro jednoduché funkce, následně definici rozšíříme na větší třídu funkcí pomocí aproximace.
\newline
\begin{definice}
Stochastický proces $S$ se nazývá \textit{jednoduchá funkce}, jestliže existuje dělení $D=\{0=t_0<t_1<\cdots<t_{n-1}<t_n=T\}$ tak, že pro každé $t$, $t_k\le t<t_{k+1}$, $k=0,1,\dots,n-1$, je $S(t,\omega)=S_k(\omega)$, pro nějaké náhodné veličiny $S_k$. \newline\newline
Trajektorie jednoduché funkce jsou po částech konstantní. 
\end{definice}


Zaveďme označení $\Delta W_k=W(t_{k+1},\omega)-W(t_k,\omega)$.
\\
\begin{definice}
Nechť $S$ je jednoduchá funkce. Pak
$$\int_0^TS\mathrm{d}W=\sum_{k=0}^{n-1}S_k\Delta W_k$$
se nazývá \textit{It\^oův integrál} funkce $S$ na intervalu $[0,T]$.
\\
\end{definice}

Pomocí limitního přechodu rozšíříme definici It\^oova integrálu pro jednoduché funkce na integrál pro obecný stochastický proces.
\begin{definice}
Nechť $\{W(t):t\ge0\}$ je Wienerův proces na pravděpodobnostním prostoru $(\Omega,\mathcal{A},\mathsf{Pr})$.
Řekneme, že stochastický proces $\{f(t,\omega):t\ge0\}$ je \textit{neanticipativní}, jestliže pro všechna $t\ge0$ hodnota $f(t,\omega)$ závisí jen na hodnotách Wienerova procesu do času $t$.
\\
\end{definice}

\begin{definice}
Nechť $W(t,\omega)$ je Wienerův proces na pravděpodobnostním prostoru $(\Omega,\mathcal{A},\mathsf{Pr})$.
Symbolem $M$ označme třídu stochastických procesů $f(t,\omega):[0,\infty)\times\Omega\to\mathbb R$ takových, že
\begin{enumerate}
\item[1.] $f(t,\omega)$ je neanticipativní, 
\item[2.] $\text{E}\left[\int_0^Tf^2(t,\omega)\mathrm{d}t\right]<\infty.$ 
\\
\end{enumerate}
\end{definice}

\begin{lemma} \label{itoint}
Nechť $f$ je náhodný proces patřící do $M$.
Pak existuje posloupnost jednoduchých funkcí $\{f_n:n\in\mathbb N\}$ tak, že pro $n\to\infty$ platí
$$\mathrm{E}\left[\int_0^T\big[f(t)-f_n(t)\big]^2\mathrm{d}t\right]\longrightarrow0.$$
\end{lemma}

\indent

\begin{definice}
Nechť $\{f_n:n\in\mathbb N\}$ je posloupnost jednoduchých funkcí z lemmatu \ref{itoint}.
Pro obecný proces $f(t,\omega)\in M$ definujeme \textit{It\^oův integrál} předpisem
$$\int_0^Tf(t,\omega)\mathrm{d}W=\lim_{n\to\infty}\int_0^Tf_n(t,\omega)\mathrm{d}W.$$
\\
\end{definice}

%%%%%%%%%%%%%%%%%%{Teorie portfolia}

\chapter{Teorie portfolia}

%%%%%%%%%%%%%%%%%%{Základní charakteristiky portfolia}

\section{Základní charakteristiky portfolia}
V teorii portfolia se investoři snaží minimalizovat investiční riziko a~zároveň maximalizovat investiční výnos. 
Avšak se zvyšováním očekáváného výnosu je spojen i~růst rizika, proto základním problémem při optimalizaci portfolia je hledání kompromisu mezi maximalizací výnosu a~minimalizací rizika spojeného s~investováním.  
V následující části popíšeme a~definujeme tyto základní charakteristiky portfolia.

\subsection{Výnosnost}
\textit{Míra výnosnosti} (nebo \textit{relativní výnosnost}) aktiva je charakteristika, která udává zisk nebo ztrátu z~investice za pevně stanovené období vyjádřená v~poměru k~množství investovaných prostředků. 
Relativní výnosnost bývá často označována pouze jako \textit{výnosnost}.
Vzhledem k~tomu, že výnosnost aktiva je pro investora nejistá (s výjimkou bezrizikového aktiva), budeme ji v~teorii portfolia chápat jako náhodnou veličinu a~budeme ji značit $r_j$.
Rozdělení pravděpodobnosti této náhodné veličiny nelze určit, nicméně v~teorii portfolia se obejdeme bez znalosti rozdělení a~využijeme pouze dále uvedené základní charakteristiky náhodné veličiny.

První z~těchto charakteristik bude střední hodnota výnosnosti aktiva, kterou označíme $\mathsf{E}(r_j)=\mu_j$.
V~teorii portfolia se ve spojitosti s~touto charakteristikou setkáváme s~pojmem \textit{očekávaná výnosnost}.
Rozptyl výnosnoti aktiva označíme $\mathsf{D}(r_j)=\sigma_j^2$.

Následující značení využijeme při definicích charakteristik porfolia.
Nechť $\boldsymbol{r}=(r_1,\dots,r_n)^\mathrm{T}$ značí náhodný vektor, jehož složky jsou výnosnosti aktiv drzěných v~portfoliu $p$.
Relativní podíly aktiv, ze kterých se skládá portfolio, se nazývají \textit{váhy} portfolia.
Vektor vah portfolia budeme značit $\boldsymbol{X}=(X_1,\dots,X_n)^\mathrm{T}$ a~přirozeně platí, že $\sum_{j=1}^nX_j=1$. 
Mezi základní charakteristiky portfolia patří jeho \textit{výnosnost}, která je dána jako $r_p=\boldsymbol{X}^\mathrm{T}\boldsymbol{r}=\sum_{j=1}^nX_jr_j$ a~je rovněž náhodnou veličinou.   
Jedním z faktorů pro výběr portfolia v Markowitzově teorii je \textit{očekávaná výnosnost portfolia}, kterou označíme $\mathsf{E}(r_p)=\mu_p$ a je zřejmé, že platí $\mu_p=\sum_{j=1}^nX_j\mu_j$.

\subsection{Riziko}
\textit{Riziko} popisuje míru nejistoty, že se skutečná výnosnost investice bude lišit od očekávané výnosnosti.  
\textit{Riziko aktiva} definujeme jako směrodatnou odchylku výnosnosti aktiva a~označíme ho $\sqrt{\mathsf{D}(r_j)}=\sigma_j$.
Analogicky je \textit{riziko portfolia} definováno jako směrodatná odchylka výnosnosti portfolia a~značeno $\sqrt{\mathsf{D}(r_p)}=\sigma_p$.  

Riziko celého portfolia v~sobě zahrnuje nejen rizika jednotlivých aktiv v~portfoliu, ale také riziko z~vzájemné závislosti výnosností jednotlivých aktiv.
Míra vzájemné závislosti dvou výnosností je mimo jiné popsána jejich kovariancí. 
Kovarianci výnosnosti aktiva~$j$ a~výnosnosti aktiva~$k$ označíme $\mathsf{C}(r_j,r_k)=\sigma_{jk}$.
Snadno se dá dokázat, že $\sigma_p=\sqrt{\sum_{j=1}^n\sum_{k=1}^nX_jX_k\sigma_{jk}}$.
Riziko portfolia je dalším faktorem pro výběr portfolia v~Markowitzově teorii.  


%%%%%%%%%%%%%%%%%%%%%%%%{Klasická teorie portfolia}


\section{Klasická teorie portfolia}

Základy klasické teorie portfolia položil Harry Markowitz svým článkem v roce 1952 \cite{markowitz}, ve kterém upozorňil na nutnost zohledňovat při výběru portfolia nejenom očekávanou výnosnost, ale i riziko změny výnosnosti.
Jeho největším přínosem bylo kvantifikování očekávaného výnosu a rizika portfolia, dále matematicky dokázal výhody diverzifikace, které byly do té doby chápány pouze intuitivně.
Markowitz zkonstruoval množinu všech dostupných portfolií v prostoru výnos-riziko, zavedl pojem \textit{efektivní množina portfolií}, ze které investor vybírá optimální portfolio pomocí indiferenčních křivek popisujících investorův vztah k riziku.

V roce 1958 rozšířil James Tobin \cite{tobin} Markowitzův model o možnost investování do bezrizikového aktiva.
\textit{Bezrizikové aktivum} má jistý výnos a tedy nulové riziko.
To má velký vliv především na efektivní množinu portfolií, která je tvořena tečnou k původní Markowitzově efektivní množině procházející bezrizikovým aktivem.
V bodě dotyku pak leží tzv. tangenciální portfolio.
Postup investora při hledání optimálního portfolia shrnuje Tobinův separační teorém, který se později vžil v souvislosti s CAPM, který zmíníme později.
Investor prvně určí tangenciální portfolio a následně ho zkombinuje s bezrizikový aktivem podle svých rizikových preferencí.

\textit{Capital Asset Pricing Model} (CAPM) vyvinuli nezávisle na sobě autoři Sharpe \cite{sharpe}, Lintner \cite{lintner} a Mossin \cite{mossin} v letech 1964-1966. CAPM zkoumá chování trhu v případě, že se všichni investoři chovají podle Markowitzovy teorie.
Zároveň vychází z Tobinova modelu, protože zahrnuje bezrizikové aktivum.
CAPM stojí na několika výchozích předpokladech, které se dají shrnout do následujících třech:
\begin{itemize}
\item[-] kapitálový trh je efektivní,
\item[-] investoři při sestavování portfolia využívají Markowitzův přístup,
\item[-] investoři mají homogenní očekávání.
\end{itemize}
Předpoklad homogenity očekávání investorů má za následek, že všichni investoři drží stejné rizikové portfolio.
Optimální portfolia investorů se liší pouze v poměru, v jakém kombinují rizikové portfolio a bezrizikové aktivum v závislosti na svých rizikových preferencích.
Toto shrnuje následující věta.

\begin{veta}(Separační teorém)
Optimální kombinace rizikových cenných papírů může být stanovena bez jakékoliv znalosti investovaných postojů k riziku a výnosnosti.
\end{veta}

Předpokládáme-li rovnováhu trhu, bývá tangenciální portfolio nazýváno jako \textit{tržní portfolio}.
Váha každého cenného papíru v tržním portfoliu je rovna jeho tržní ceně.

%%%%%%%%%%%%%%%%%%%%%%%%%%{Stochastická teorie portfolia}

\subsection{Stochastická teorie portfolia}
V této podkapitole jsou shrnuty některé základní poznatky Stochastické teorie portfolia, která byla inspirací pro studium spojitých procesů v teorii portfolia uvedených v této práci.
Budeme diskutovat o výhodách a nevýhodách tohoto přístupu.

Pojem \textit{Stochastická teorie portfolia} (SPT) zavedl Robert Fernholz ve své monografii \cite{fern}.
Jedná se o matematickou teorii, která zkoumá chování portfolia jakožto i strukturu kapitálového trhu.
Hlavní výhodou oproti klasické teorii portfolia jsou méně  striktní předpoklady, které dovolují konzistentní přístup bez výskytu Ohlson-Rosenbergova paradoxu \cite{ohlson}.

Na rozdíl od Mertonova modelu se v SPT nepředpokládá tržní rovnováha.
Parametry stochastického procesu, kterým se řídí cena aktiva, jsou v SPT také stochastické procesy nikoli konstanty jako v Mertonově přístupu.
Dokonce, na rozdíl od klasického přístupu, se v SPT neklade důraz na předpoklad neexistence tržní arbitráže, přičemž jsou studovány vlastnosti trhu vedoucí k existenci arbitráže. 

Pro vývoj ceny aktiva Fernholz používá následující spojitý logaritmický model
$$\mathrm{d}\log P_j(t)=\gamma_j(t)\mathrm{d}t+\sum_{k=1}^{n}\xi_{jk}(t)\mathrm{d}W_k(t),$$
kde $P_j(t)$ je cena aktiva $j$ v čase $t$, $\gamma_j(t)$ a $\xi_{jk}(t)$ jsou stochastické procesy a $\boldsymbol{W}(t)=(W_1(t),\dots,W_n(t))^\mathrm{T}$ je $n$-dimenzionální Wienerův proces. 
Proces $\gamma_j(t)$ se nazývá \textit{míra růstu} a $\xi_{jk}(t)$ se nazývá proces \textit{volatility}.
Míru růstu v logaritmickém modelu je možno odvodit z míry výnosnosti ve standardním modelu a naopak podle následujícího vztahu
$$\mu_j(t)=\gamma_j(t)+\frac12\sum_{k=1}^{n}{\xi_{jk}}^2(t).$$
Robert Fernholz uvádí, že v dlouhém časovém horizontu je chování hodnoty portfolia lépe popsáno mírou růstu než mírou výnosnosti.
Uvažujeme-li kovarianční matici $\boldsymbol{\Sigma}$ výnosností aktiv, pak matice volatilit $\boldsymbol{\xi}$ je definována jako
$$\boldsymbol{\Sigma}=\boldsymbol{\xi}\boldsymbol{\xi}^\mathrm{T}.$$
Tedy $\boldsymbol{\xi}$ je maticová odmocnina z $\boldsymbol{\Sigma}$.

Jedním ze základních pojmů v SPT jsou \textit{portfolio generující funkce}, pomocí nichž jsou tvořeny portfolia, které mají dobře definované výnosnosti. Podrobnější informace je možno najít v knize \cite{fern}.



%%%%%%%%%%%%%%%%%%%%%%%%%%%%%%%

\section{Paradox v teorii portfolia}
Nekonzistencí obvyklých předpokládů teorie portfolia se v roce 2009 zabýval také John Stalker profesor na univerzitě v Princetonu. 
V preprintu \cite{john} uvádí důkaz, kde využívá separačního theorému pro dva fondy. Základní myšlenky tohoto důkazu shrnujeme v následujícím textu.

Uvažujme následující předpoklady obvyklé pro teorii portfolia.
\begin{enumerate}
\item Ceny podkladových aktiv se řídí It\^oovými procesy s konstantními očekávanými mírami výnosnosti a konstantními kovariancemi výnosností. 
\item Neexistují žádná omezení na množství aktiv držených investorem. Aktiva jsou navíc nekonečně dělitelná.
\item Neexistují žádné transakční náklady a cena prodeje a nákupu podkladového aktiva se neliší.
\item 
\item
\end{enumerate}

\begin{displaymath}
\Delta p_i = \frac{a_i}{n_i}\Delta A + \frac{b_i}{n_i}\Delta B
\end{displaymath}
 and the return on
the $i$'th asset  is
\begin{displaymath}
\frac{a_i}{n_ip_i(t)}\Delta A + \frac{b_i}{n_ip_i(t)}\Delta B
\end{displaymath}
where $\Delta A = A(t+\Delta t) - A(t)$ and $\Delta B = B(t+\Delta t)
- B(t)$.  It follows that the covariance rate matrix for the returns
can be factored as
$$
%\begin{displaymath}
\begin{pmatrix}
\frac{a_1}{n_1p_1} & \frac{b_1}{n_1p_1}\cr \vdots & \vdots\cr \frac{a_m}{n_mp_m} & \frac{b_m}{n_mp_m}
\end{pmatrix}
%\end{displaymath}
\times
%\begin{displaymath*}
\begin{pmatrix}
\frac{\mathsf{C}(\Delta A, \Delta A)}{\Delta t} &
\frac{\mathsf{C}(\Delta A, \Delta B)}{\Delta t} \cr
\frac{\mathsf{C}(\Delta B, \Delta A)}{\Delta t} & \frac{\mathsf{C}(\Delta B,
\Delta B)}{\Delta t} 
\end{pmatrix}
%\end{displaymath*}
\times
%\begin{displaymath*}
\begin{pmatrix}
\frac{a_1}{n_1p_1} & \ldots & \frac{a_m}{n_mp_m}\cr \cr \frac{b_1}{n_1p_1} & \ldots &
\frac{b_m}{n_mp_m}.
\end{pmatrix}
%\end{displaymath*}
$$


%%%%%%%%%%%%%%%%%%%%%%%%%%%%%%%


\nocite{*}  %umisti do literatury i necitovanou polozku a \nocite{*} umisti do literatury vsechny polozky z bibtex databaze 

\addcontentsline{toc}{chapter}{Literatura}
%\bibliographystyle{plain}
\bibliographystyle{abbrv}
\bibliography{bibliography}

\end{document}
