 \documentclass[a4paper,12pt]{report}
  \pagestyle{plain}
  \usepackage{latexsym}      %symboly
  \usepackage{graphicx}
  \usepackage{amsmath}
  \usepackage{amsfonts}
  \usepackage{amsthm}
  \usepackage{epstopdf} %pokud pouzivam .eps obrazky, ale chci prekladat pdflatexem 
  \usepackage[czech]{babel}
  \usepackage[utf8]{inputenc}
  \usepackage{comment}

\usepackage{url}  %kvuli vkladani URL adres do bibtexu ale asi jen ve stylu CZECHISO
\DeclareUrlCommand\url{\def\UrlLeft{<}\def\UrlRight{>} \urlstyle{tt}}
  
%\renewcommand{\figurename}{Fig.}
  \usepackage{phdthesis}
  \usepackage{psfrag}
%  \newtheorem{definice}{Definition}
  \usepackage{hyperref} % klikatelne odkazy v textu
  \hypersetup{
      colorlinks   = true, %Colours links instead of ugly boxes
      urlcolor     = blue, %Colour for external hyperlinks
      linkcolor    = red, %Colour of internal links
      citecolor   = red %Colour of citations
    }
  \date{\today}
  \usepackage{minibox}
  \usepackage{pdfpages}
%
%
\vfuzz10pt % Don't report over-full v-boxes if over-edge is small
\hfuzz10pt
%
%
%\newtheorem{theorem}{Theorem}[chapter]
%\newtheorem{definition}{Definition}[chapter]
%\newtheorem{example}{Example}[chapter]
\DeclareMathOperator{\limean}{l.i.m.}   %limit in the mean se stane operatorem
\DeclareMathOperator{\argmax}{argmax}   %stejne tak argmin a argmax
\DeclareMathOperator{\argmin}{argmin}
%
%

\newtheorem{veta}{Věta} 
\newtheorem{lemma}[veta]{Lemma}
\newtheorem{dusledek}[veta]{Důsledek}
\theoremstyle{definition} \newtheorem{definice}[veta]{Definice}
\newtheorem{priklad}{Příklad}
\theoremstyle{remark}
\newtheorem*{poznamka}{Poznámka}
\newtheorem*{reseni}{Řešení}  



\newcommand{\specialcell}[2][c]{%
\begin{tabular}[#1]{@{}l@{}}#2\end{tabular}}



\begin{document}
\titlepage
\vspace*{-2.4cm}
\begin{center}
\textsc{\Large{Masarykova Univerzita}} \\
\large{Přírodovědecká fakulta} \\
\large{Ústav matematiky a statistiky}
\end{center}
\vspace{7.11cm}
%
\begin{center}
\textsc{\Large{DISERTAČNÍ PRÁCE}}
\end{center}
%
\vspace{11.39cm}
%
\begin{center}
\begin{tabular}{l  r}
 \Large{Brno 2015} \qquad\qquad\qquad\qquad \qquad\qquad & \Large{Lenka Křivánková}
\end{tabular}
\end{center}

\titlepage
\begin{center}
\textsc{\Large{Masarykova Univerzita}} \\
\textsc{\Large{Přírodovědecká fakulta}} \\
\textsc{\large{Ústav matematiky a statistiky}}
\end{center}\quad \\
%
\begin{center}
\includegraphics[width=5cm]{IMG/sci-logo.pdf}
\end{center}\quad \\
%
\begin{center}
\textbf{\Large{Stochastické metody analýzy ekonomických dat}} \\ \quad \\
\textsc{\large{Disertační práce}}
\end{center} \quad \\\\\\\\\\\\\\\\\\\\\\\\\\ \\
\begin{center}
\large{
\begin{tabular}{l  r}
\Large{Brno 2015} \qquad\qquad\qquad\qquad \qquad\qquad & \Large{Lenka Křivánková}
\end{tabular}
}
\end{center} \quad
%
%\begin{center}
%\large{Brno 2009}
%\end{center}
%\begin{center}
%\verb=   (\(\=\\
%\verb=  (._. )=\\
%\verb=(')(')_)o=
%\end{center}

\normalsize



\chapter*{Bibliografický záznam}\addcontentsline{toc}{chapter}{Bibliografický záznam}
\setcounter{page}{1}
\pagenumbering{roman}
%\thispagestyle{empty}

\begin{tabular}{p{3.9cm}  p{9.1cm}}
\textbf{Autor:} & Mgr. Lenka Křivánková \\
\textbf{Název:} & Stochastické metody analýzy ekonomických dat \\
\textbf{Školitel:} & doc. RNDr. Martin Kolář, Ph.D.  \\
\textbf{Studijní program:} & Matematika \\
\textbf{Obor:} & Pravděpodobnost, statistika a matematické modelování \\
\textbf{Rok obhajoby:} & 2015 \\
\textbf{Klíčová slova:} & Stochastické procesy, \\
\end{tabular}
\\\\\\\\\\

\begin{flushright} {{\Huge Bibliographic entry}} \vspace{38pt} \end{flushright}
%\addcontentsline{toc}{chapter}{Bibliographic entry}

\hspace{-0.7cm}
\begin{tabular}{p{3.9cm}  p{9.1cm}}
\textbf{Author:} & Mgr. Lenka Křivánková \\
\textbf{Title:} & Stochastic methods in analysis of economic data \\
\textbf{Supervisor:} & doc. RNDr. Martin Kolář, Ph.D.  \\
\textbf{Study programme:} & Mathematics \\
\textbf{Study field:} & Probability, statistics and mathematical modeling \\
\textbf{Year of defence:} & 2015 \\
\textbf{Keywords:} & Stochastic processes, \\
\end{tabular}
\newpage
\pagestyle{empty}
\null
\vfill
\begin{center}
\copyright \quad Lenka Křivánková, Masarykova Univerzita, 2015
\end{center}

\chapter*{Poděkování}\addcontentsline{toc}{chapter}{Poděkování}
%\thispagestyle{empty}
\pagestyle{plain}
Na tomto místě
\\

Brno, Listopad 2015
\\

Lenka Křivánková

\chapter*{Abstrakt}\addcontentsline{toc}{chapter}{Abstrakt}
%\thispagestyle{empty}
V první části práce jsou uvedeny základní matematické pojmy použité v dalších částech textu. 
\\\\\\\\\\

\begin{flushright} {{\Huge Abstract}} \vspace{38pt} \end{flushright}
%\addcontentsline{toc}{chapter}{Abstract}
In the first part of the work, some basic mathematical methods employed in this thesis are recalled. 

\tableofcontents \addcontentsline{toc}{chapter}{Obsah}

\chapter*{Seznam použitého značení}\addcontentsline{toc}{chapter}{Seznam použitého značení} 
%\hspace{-0.7cm}
%\begin{tabular}{r  l}
%$\mathsf{E}X$ & Expected value of random variable $X$ \\
%$\mathsf{var}X$ & Dispersion, resp. variance of random variable X \\
%\end{tabular}

\normalsize
\textbf{Pravděpodobnost}\\\\
%  \begin{table}[h]
   \begin{tabular}{p{4cm} p{9.3cm}}
   $\omega$				        & elementární jev\\
   $\Omega$                                          &   základní prostor, množina všech elementárních jevů \\
   $\mathsf{Pr}(A)$                               &  pravděpodobnost jevu $A$ \\
   $(\Omega,\mathcal{A}, \mathsf{Pr})$                         &   pravděpodobnostní prostor \\
   \end{tabular}\\\\\\
%
%  \end{table}
%
%
\textbf{Náhodné veličiny}\\\\
%  \begin{table}[h]
   \begin{tabular}{p{4cm} p{9.3cm}}
   $X$                                                &   náhodná veličina \\
   $f(x)$; $f_X(x)$                                               &   hustota pravděpodobnosti náhodné veličiny $X$ \\ %probability density of the random variable $X$ \\
   $F(x)$; $F_X(x)$				 &  distribuční funkce náhodné veličiny $X$ \\
%   $f(x; \theta)$                                    &   probability density function of random variable $X$ with distribution dependent on scalar parameter $\theta$ \\
%   $f(x,t)$                                          &    probability density of random process $X(t)$ at fixed time $t$ \\
%   $f(x,t | x_0, t_0)$                               &    probability density of random process $X(t)$ at fixed time $t$ conditioned by its state $x_0$ at time $t_0$ \\
   $X\sim \mathcal{L}$                                &   náhodná veličina $X$ mající rozdělením pravděpodobnosti $\mathcal{L}$ \\ %random variable $X$ having general probability distribution $\mathcal{L}$\\
   $\mathsf{E}(X)$;  $\mathsf{E}X$                         	      &   střední hodnota náhodné veličiny $X$ \\
   $\mathsf{D}(X)$;  $\mathsf{D}X$                       	      &   rozptyl náhodné veličiny $X$ \\
   $\mathsf{C}(X_i,X_j)$             		      &   kovariance náhodných veličin $X_i$ a $X_j$\\
   $\mathrm{d}X(t,\omega)$; $\mathrm{d}X(t)$; $\mathrm{d}X_t$    &  stochastický diferenciál \\ %continuous-time random process \\
   $X(t,\omega)$; $X(t)$; $X_t$                       &   stochastický proces \\
   $W(t,\omega)$; $W(t)$; $W_t$                       &   standardní Wienerův proces \\
   $\boldsymbol{\Delta} W_t$			                      &   přírůstek Wienerova procesu \\
   $\mathsf{N}(\mu, \sigma^{2})$                      &   Normální (Gaussovo) rozdělení pravděpodobnosti se střední hodnotou $\mu$ a rozptylem $\sigma^2$  \\
   \end{tabular}\\\\\\
%
%
\textbf{Náhodné vektory}\\\\
%  \begin{table}[h]
   \begin{tabular}{p{4cm} p{9.3cm}}
  $\boldsymbol{X}=(X_1,\dots,X_n)^\mathrm{T}$                 &     $n$-rozměrný náhodný vektor \\
  $f(\mathbf{x})$; $f_{\boldsymbol{X}}(\mathbf{x})$                                             &   sdružená hustota pravděpodobnosti náhodného vektoru  $\boldsymbol{X}$\\
   $F(\mathbf{x})$; $F_{\boldsymbol{X}}(\mathbf{x})$                                             &   sdružená distribuční funkce náhodného vektoru  $\boldsymbol{X}$\\
%   $\boldsymbol{\theta}=(\theta_1,\ldots,\theta_m)$  &    m-dimensional vector parameter \\
%   $f(\mathbf{x};\boldsymbol{\theta})$               &    probability density of the random vector $\boldsymbol{X}$ with vector parameter $\boldsymbol{\theta}$ \\
    $\mathsf{E}(\boldsymbol{X})$; $\mathsf{E}\boldsymbol{X}$                             &   vektor středních hodnot náhodného vektoru $\boldsymbol{X}$\\
   $\mathsf{D}(\boldsymbol{X})$; $\mathsf{D}\boldsymbol{X}$                             &   kovarianční matice náhodného vektoru $\boldsymbol{X}$\\
    $\boldsymbol{\Sigma}(\boldsymbol{X_i},\boldsymbol{X_j})$     &   kovarianční matice náhodných vektorů $\boldsymbol{X_i}$ a $\boldsymbol{X_j}$\\
$\boldsymbol{X}(\boldsymbol{t},\boldsymbol{\omega})$; $\boldsymbol{X}(\boldsymbol{t})$; $\boldsymbol{X}_{\boldsymbol{t}}$  				& vektorový stochastický proces\\
$\boldsymbol{W}(\boldsymbol{t})$ 		& standardní $n$-rozměrný Wienerův proces\\
   \end{tabular}\\\\\\
%
%
%
\\\\\\\
\textbf{Prostory a matice}\\\\
%  \begin{table}[h]
   \begin{tabular}{p{4cm} p{9.3cm}}
   $\mathbb{R}$                              &   jednorozměrný Eukleidovský prostor \\
   $\mathbb{R}^n$                              &     $n$-rozměrný Eukleidovský prostor \\
   $\boldsymbol{x}=(x_1,\ldots,x_n)^\mathrm{T}$             &    $n$-rozměrný reálný vektor \\
    $\boldsymbol{1}$                           &   vektor jedniček\\
   $\mathbf{I}_n$                              &    jednotková matice řádu $n$ \\
   $\mathbf{A}^\mathrm{T}$                              &   transponovaná matice vzhledem k matici $\mathbf{A}$\\
   \end{tabular}\\\\\\
% \end{table}
%
%
\textbf{Teorie portfolia}\\\\
%  \begin{table}[h]
   \begin{tabular}{p{4cm} p{9.3cm}}
   $I_t$ 								& množina všech investorů na trhu v čase $t$\\
   $P_j(t)$                             &   cena aktiva $j$ v čase $t$ \\
   $\boldsymbol{P}(t)$                             &   vektor cen aktiv v čase $t$\\
   $P_j(t)$                             &   objem aktiv $j$ v čase $t$ \\
   $\boldsymbol{N}(t)$     &   vektor podílů aktiv v čase $t$\\
   $V_j(t)=P_j(t)N_j(t)$       &  tržní hodnota aktiva $j$\\
   $r_j$  								& míra výnosnosti podkladového aktiva $j$\\
   $\boldsymbol{r}=(r_1,\dots,r_n)^\mathrm{T}$  		& vektor výnosností aktiv držených v~portfoliu\\
   $\mu_j$ 								& očekávaná míra výnosnosti podkladového aktiva\\
   $\sigma_j$								& riziko podkladového aktiva\\
   $B(t)$ 								& cena bezrizikového aktiva v čase $t$\\
   $r_f(t)$ 								& bezriziková úroková míra v čase $t$\\
   $\boldsymbol{X}=(X_1,\dots,X_n)^\mathrm{T}$		& vektor vah podkladových aktiv držených v~portfoliu\\
   $\mathsf{C}(r_j,r_k)=\sigma_{jk}$				& kovariance výnosnosti aktiva~$j$ a~výnosnosti aktiva~$k$\\
   $\boldsymbol{\Sigma}$						& kovarianční matice výnosností podkladových aktiv držených v~portfoliu\\
   $\xi_{jk}(t)$ 							& proces volatility podkladových aktiv držených v~portfoliu\\
   $\boldsymbol{\xi}$						& matice volatilit podkladových aktiv držených v~portfoliu\\
   $\gamma_j(t)$ 							& míra růstu  podkladového aktiva $j$\\
   $w_{j}(i,t)$ 							& optimální objem prostředků investovaných do aktiva $j$ investorem $i$ v čase $t$\\
   \end{tabular}\\\\\\
%
%
%


\chapter{Úvod}
\pagestyle{plain}
\setcounter{page}{1}
\pagenumbering{arabic}

\begin{comment}
Klasická kniha o Stochastických diferenciálních rovnicích \cite{oksendal2003stochastic}\\%
Numerické řešení stochastických diferenciálních rovnic a jejich soustav \cite{kloeden1997}\\
Vzorec pro váhy tržního portfolia čerpáme z \cite{fabozzi}\\
Základy teorie portfolia pokládá Markowitz v \cite{markowitz}\\%
Tobin v \cite{tobin} rozšiřuje Markowitzův přístup o bezrizikové aktivum.%
Capital Asset Pricing Model (CAPM), který nezávisle na sobě zkonstruovali Sharpe (1964) v \cite{sharpe1964}, Lintner (1965) v \cite{lintner1965} a Mossin (1966) v \cite{mossin1966} zkoumá chování trhu v případě, že se všichni investoři chovají podle Markowitzovy teorie.%
První řešení spojitého modelu v teorii portfolia dává Merton v \cite{merton1971}\\%
Na rozpor předpokladů v Mertonově článku \cite{merton1971} upozorňují Ohlson a Rosenberg v \cite{ohlson}\\%
Na jejich článek reaguje Merton v \cite{merton1975} a jeho reakce vychází ješte před článkem Ohlsona a Rosenberga.\\%
\\
Konzistentní dynamický přístup k teorii portfolia dává Fernholz ve své knize Stochastická teorie portfolia \cite{fern} a v článku s Karatzasem\cite{kara}\\
\\
Stochastický kalkul, oceňování finančních derivátů, arbitrážní teorie - Pascucci\cite{pascucci}\\%
\\ \\
Continuous asset model \cite{higham2004introduction}\\
Riziková averze, vše o portfoliu, separační teorém, objemný \cite{ingersoll1987theory}\\
Dynamic portfolio optimization \cite{prigent2007portfolio}\\
General Equilibrium Asset Pricing \cite{wickens2012macroeconomic}\\
Arbitráž, asset pricing model, numerické metody \cite{wilmott1995mathematics}\\
Teorie portfolia \cite{de2010portfolio}, \cite{brada}, \cite{sharpe}, úvod \cite{bohdalova}
Diferenciální rovnice, numerické metody \cite{cyganowski1998maple}, \cite{cyganowski2002elementary}\\
Stochastické procesy \cite{allen2010introduction} \\%
Srochasticke procesy a rovnováha na trhu\cite{karatzas1998methods} \\
Numerické řešená SDE ve financich \cite{sauer2012numerical} \\


Finanční trhy, modely volatility, optimální portfolio, opce \cite{bouchaud2003theory} \\
Black-Scholes, Arbitraz, opce, numericke metody \cite{duffie2010dynamic} \\
arbitraz, Ito, spojite modely, stochasticka volatilita \cite{duffie2003intertemporal} \\
mnohorozmerny model ceny aktiva. \cite{etheridge2002course} \\%
vse o portfoliu \cite{fabozziportfolio} \\
Na optimalizaci portfolia pomocí bayesovských metod se specializuje v článku \cite{fabozzi2007robust}\\
modely volatility \cite{gali2009monetary} \\
ocekavana vynosnost, arbitrazni teorie, konstrukce portfolia \cite{gridold1999active} \\
Wienerův proces, volatilita, základní numerické metody, VaR \cite{hull} \\
Základy finanční matematiky a teorie portfolia \cite{melichercik} \\
Dynamická teorie portfolia, \cite{oberuc2003dynamic}\\
Úvod do SDE \cite{karatzas2012brownian}, \cite{gard}\\%
Stochastická analýza \cite{shreve2012stochastic} a \cite{shreve2004stochastic}\\%

\end{comment}

\chapter{Stochastická analýza}
%\chapter{Teoretická východiska}
%\begin{comment}
Nejprve uvedeme základní matematický aparát, který budeme využívat v této práci.
Budeme používat obvyklé značení, jehož přehled je uveden na začátku práce. 
Zavedeme základní definice a vztahy využívané ve stochastickém modelování.
Tato kapitola čerpá především z klasické monografie Bernta {\O}ksendala o stochastických diferenciálních rovnicích \cite{oksendal2003stochastic}, 
podobně jako knihy Thomase Garda  \cite{gard} 
a rozsáhlé publikace Andrei Pascucci zabívající se matematickými metodami oceňování opcí \cite{pascucci}.
Jako další reference byly využity \cite{karatzas2012brownian}, \cite{allen2010introduction},  \cite{shreve2012stochastic} a \cite{shreve2004stochastic}.
Ve zmiňované literatuře je možno nalézt další podrobnosti.

\section{Stochastické procesy a jejich vlastnosti}
V první části definujeme stochastický proces a některé jeho zajímavé vlastnosti.

%\subsection{Stochastický proces}
\begin{definice}[Stochastický proces]
Uvažujme  indexovou množinu $T$ a prav\-dě\-po\-dob\-nostní prostor $(\Omega,\mathcal{A}, \mathsf{Pr})$.
\textit{Stochastický proces} $X(t,\omega)$ je funkce dvou proměnných $X:T\times\Omega\to\mathbb{R}$, kde
\begin{itemize}
\item $X(t,\cdot):\Omega\to\mathbb{R}$ je pro každé $t\in T$ náhodná veličina,
\item $X(\cdot,\omega):T\to\mathbb{R}$ je pro každé $\omega\in\Omega$ realizace náhodného procesu.
\end{itemize}
Reálná funkce $X(\cdot,\omega)$ se také nazývá \textit{trajektorie} nebo \textit{cesta} stochastického procesu. 
\end{definice}
Množina parametrů $T$ může být diskrétní nebo spojitá a v závislosti na tom pak mluvíme o \textit{diskrétním} nebo \textit{spojitém} stochastickém procesu. 

\begin{definice}[Ekvivalence stochastických procesů]
Dva stochastické procesy $X(t,\omega)$ a $Y(t,\omega)$ nazveme \textit{ekvivalentní} pokud platí
$$X(t,\cdot)=Y(t,\cdot)\text{ s pravděpodobností jedna pro každé }t\in T.$$ 
V takovém případě hovoříme o tom, že jeden proces je \textit{verzí} druhého procesu.
\end{definice}

\begin{definice}[Systém distribučních funkcí stochastického procesu]
Nechť $T_n$ je množina všech vektorů $T_n=\{\boldsymbol{t}=(t_1,\dots,t_n):t_1\leq t_2\leq\cdots\leq t_n; t_i\in T; i=1,\dots,n\}$,
pak \textit{distribuční funkcí} stochastického procesu rozumíme funkci
$$F_{\boldsymbol{t}}(\boldsymbol{x})=F_{t_1,\dots,t_n}(x_1,\dots,x_n)= \mathsf{Pr}(X_{t_1}\leq x_1,\dots,X_{t_n}\leq x_n),$$
pro všechna $\boldsymbol{t}=(t_1,\dots,t_n)\in T$ a všechna $\boldsymbol{x}=(x_1,\dots,x_n)\in \mathbb{R}^n$.
Pro různá $n$ a různé $t_1,\dots,t_n$ dostáváme celý \textit{systém distribučních funkcí}, který označíme $\mathcal{F}$.
\end{definice}

\begin{definice}[Stacionarita stochastického procesu]
Stochastický proces $X(t)$ nazveme \textit{striktně stacionární} pokud jsou jeho sdružené distribuční funkce invariantní vůči časovému posunu, tedy pro všechna $h\in\mathbb{R}$ takové, že $t_j\in T$ a $t_j+h\in T$ pro všechna $j$ platí
$$F_{t_1+h,\dots,t_n+h}(x_1,\dots,x_n)=F_{t_1,\dots,t_n}(x_1,\dots,x_n).$$   

Stochastický proces $X(t)$ nazveme \textit{slabě stacionární} pokud existuje konstanta $\mu\in\mathbb{R}$ a funkce $c:\mathbb{R}^+\to\mathbb{R}$ taková, že
$$\mathsf{E}{X(t)}=\mu\quad\text{ a }\quad\mathsf{C}(X(t),X(s))=c(t-s)$$
pro všechna $t,s\in T$.
\end{definice}

%\begin{definice}[]
%\end{definice}


\subsection{Wienerův proces}

\begin{definice}
Reálný stochastický proces $\{W(t):t\ge0\}$ na pravdě\-podob\-nostním prostoru $(\Omega,\mathcal{A},\mathsf{Pr})$ se nazývá \textit{Wienerův proces}, jestliže platí
\begin{enumerate}
\item[1.]$W(0)=0$, 
\item[2.](spojitost trajektorií Wienerova procesu) s pravděpodobností jedna je funkce $t\to W(t)$ spojitá v $t$,
\item[3.](nezávislost a stacionarita přírůstků) přírůstky procesu jsou nezávislé a stacionární, t.j.  pro každé $t\ge s\ge0$ má $X(t+s)-X(s)$ stejné rozdělení jako $X(t)-X(0)$,
\item[4.](normalita přírůstků) pro každé $t\ge s\ge0$ má přírůstek $W(t)-W(s)$ normální rozdělení $\mathsf{N}(0, t-s)$.
\end{enumerate}
\end{definice}

\subsection{Wienerův proces s driftem}
\begin{definice}
Stochastický proces $X(t)$ definovaný 
$$X(t)  = \mu t + \sigma W (t)$$
se nazývá \textit{Wienerův proces  s driftem} $\mu$ a volatilitou $\sigma$.
\end{definice}

\subsection{It\^oův proces}

\begin{definice}
Nechť $W(t,\omega)$ je Wienerův proces na pravděpodobnostním prostoru $(\Omega,\mathcal{A},\mathsf{Pr})$.
\textit{It\^oův proces} je stochastický proces tvaru
$$X(t,\omega)=\int_0^tU(s,\omega)\mathrm{d}s+\int_0^tV(s,\omega)\mathrm{d}W(s,\omega),$$
kde $U(t,\omega)$ a $V(t,\omega)$ jsou stochastické procesy patřící do $M$.

Často se používá diferenciální tvar, nazývaný \textit{stochastický diferenciál},
$$\mathrm{d}X(t,\omega)=U(t,\omega)\mathrm{d}t+V(t,\omega)\mathrm{d}W(t,\omega).$$
\end{definice}


\subsection{Geometrický Wienerův proces}
\begin{definice}
Stochastický proces $S(t)$ definovaný 
$$\frac{\mathrm{d} S(t)}{S(t)} = \mu\mathrm{d}t + \sigma\mathrm{d}W (t)$$
se nazývá \textit{geometrický Wienerův proces}.
\end{definice}

\subsection{It\^oovo lemma}
It\^oovo lemma je nástrojem pro práci s It\^oovým integrálem a stochastickými diferenciálními rovnicemi.
It\^oovo lemma říká, že It\^oovy procesy tvoří uzav\-ře\-nou třídu vzhledem ke skládání s hladkými funkcemi.

\begin{veta}[It\^oovo lemma]
Nechť $X(t,\omega)$ je It\^oův proces se stochastickým diferenciálem
$$\mathrm{d}X=U\mathrm{d}t+V\mathrm{d}W.$$
Nechť $g(t,x):(0,\infty)\times\mathbb R\to\mathbb R$ je dvakrát spojitě diferencovatelná funkce.
Potom $Y(t)=g(t,X(t))$ je také It\^oův proces.
Jeho stochastický diferenciál má tvar
\begin{alignat*}{2}
\mathrm{d}Y=\frac{\partial g}{\partial t}\mathrm{d}t+\frac{\partial g}{\partial x}\,\mathrm{d}X+\frac12\frac{\partial^2 g}{\partial x^2}(\mathrm{d}X)^2=\left[\frac{\partial g}{\partial t}+\frac{\partial g}{\partial x}U+\frac12\frac{\partial^2 g}{\partial x^2}V^2\right]\mathrm{d}t+\frac{\partial g}{\partial x}V\mathrm{d}W.
\end{alignat*}
\end{veta}

\subsection{It\^oův integrál}
Důležitým nástrojem pro počítání stochastických diferenciálních rovnic je It\^oův integrál.
Budeme ho definovat obdobným způsobem jako Riemannův integrál.
Nejprve It\^oův integrál definujeme pro jednoduché funkce, následně definici rozšíříme na větší třídu funkcí pomocí aproximace.
\newline
\begin{definice}
Stochastický proces $S$ se nazývá \textit{jednoduchá funkce}, jestliže existuje dělení $D=\{0=t_0<t_1<\cdots<t_{n-1}<t_n=T\}$ tak, že pro každé $t$, $t_k\le t<t_{k+1}$, $k=0,1,\dots,n-1$, je $S(t,\omega)=S_k(\omega)$, pro nějaké náhodné veličiny $S_k$. \newline\newline
Trajektorie jednoduché funkce jsou po částech konstantní. 
\end{definice}


Zaveďme označení $\Delta W_k=W(t_{k+1},\omega)-W(t_k,\omega)$.
\\
\begin{definice}
Nechť $S$ je jednoduchá funkce. Pak
$$\int_0^TS\mathrm{d}W=\sum_{k=0}^{n-1}S_k\Delta W_k$$
se nazývá \textit{It\^oův integrál} funkce $S$ na intervalu $[0,T]$.
\\
\end{definice}

Pomocí limitního přechodu rozšíříme definici It\^oova integrálu pro jednoduché funkce na integrál pro obecný stochastický proces.
\begin{definice}
Nechť $\{W(t):t\ge0\}$ je Wienerův proces na pravdě\-podob\-nostním prostoru $(\Omega,\mathcal{A},\mathsf{Pr})$.
Řekneme, že stochastický proces $\{f(t,\omega):t\ge0\}$ je \textit{neanticipativní}, jestliže pro všechna $t\ge0$ hodnota $f(t,\omega)$ závisí jen na hodnotách Wienerova procesu do času $t$.
\\
\end{definice}

\begin{definice}
Nechť $W(t,\omega)$ je Wienerův proces na pravděpodobnostním prostoru $(\Omega,\mathcal{A},\mathsf{Pr})$.
Symbolem $M$ označme třídu stochastických procesů $f(t,\omega):[0,\infty)\times\Omega\to\mathbb R$ takových, že
\begin{enumerate}
\item[1.] $f(t,\omega)$ je neanticipativní, 
\item[2.] $\text{E}\left[\int_0^Tf^2(t,\omega)\mathrm{d}t\right]<\infty.$ 
\\
\end{enumerate}
\end{definice}

\begin{lemma} \label{itoint}
Nechť $f$ je náhodný proces patřící do $M$.
Pak existuje posloupnost jednoduchých funkcí $\{f_n:n\in\mathbb N\}$ tak, že pro $n\to\infty$ platí
$$\mathrm{E}\left[\int_0^T\big[f(t)-f_n(t)\big]^2\mathrm{d}t\right]\longrightarrow0.$$
\end{lemma}

\indent

\begin{definice}
Nechť $\{f_n:n\in\mathbb N\}$ je posloupnost jednoduchých funkcí z lemmatu \ref{itoint}.
Pro obecný proces $f(t,\omega)\in M$ definujeme \textit{It\^oův integrál} předpisem
$$\int_0^Tf(t,\omega)\mathrm{d}W=\lim_{n\to\infty}\int_0^Tf_n(t,\omega)\mathrm{d}W.$$
\\
\end{definice}
%\end{comment}
%%%%%%%%%%%%%%%%%%{Teorie portfolia}

\chapter{Teorie portfolia}

%%%%%%%%%%%%%%%%%%{Základní charakteristiky portfolia}

\section{Základní charakteristiky portfolia}
V teorii portfolia se investoři snaží minimalizovat investiční riziko a~zároveň maximalizovat investiční výnos. 
Avšak se zvyšováním očekávaného výnosu je spojen i~růst rizika, proto základním problémem při optimalizaci portfolia je hledání kompromisu mezi maximalizací výnosu a~minimalizací rizika spojeného s~investováním.  
V následující části popíšeme a~definujeme tyto základní charakteristiky portfolia.

\subsection{Výnosnost}
\textit{Míra výnosnosti} (nebo \textit{relativní výnosnost}) aktiva je charakteristika, která udává zisk nebo ztrátu z~investice za pevně stanovené období vyjádřená v~poměru k~množství investovaných prostředků.

\begin{definice}
Investujeme-li $Y$ korun do aktiva $j$ v čase $t$, pak korunová hodnota investice v čase $t+\Delta t$ bude $[1+r_j(t,t+\Delta t)]Y$, kde  $r_j(t,t+\Delta t)$ definujeme jako \textit{míru výnosnosti}.  
\end{definice}
 
Relativní výnosnost bývá často označována pouze jako \textit{výnosnost}.
Vzhledem k~tomu, že výnosnost aktiva je pro investora nejistá (s výjimkou bezrizikového aktiva), budeme ji v~teorii portfolia chápat jako náhodnou veličinu. % a~budeme ji značit $r_j$.
Rozdělení pravděpodobnosti této náhodné veličiny nelze určit, nicméně v~teorii portfolia se obejdeme bez znalosti rozdělení a~využijeme pouze dále uvedené základní charakteristiky náhodné veličiny.

První z~těchto charakteristik bude střední hodnota výnosnosti aktiva, kterou označíme $\mathsf{E}(r_j)=\mu_j$.
V~teorii portfolia se ve spojitosti s~touto charakteristikou setkáváme s~pojmem \textit{očekávaná výnosnost}.
Rozptyl výnosnosti aktiva označíme $\mathsf{D}(r_j)=\sigma_j^2$.

Následující značení využijeme při definicích charakteristik porfolia.
Nechť $\boldsymbol{r}=(r_1,\dots,r_n)^\mathrm{T}$ značí náhodný vektor, jehož složky jsou výnosnosti aktiv držených v~portfoliu $p$.
Relativní podíly aktiv, ze kterých se skládá portfolio, se nazývají \textit{váhy} portfolia.
Vektor vah portfolia budeme značit $\boldsymbol{X}=(X_1,\dots,X_n)^\mathrm{T}$ a~přirozeně platí, že $\sum_{j=1}^nX_j=1$. 
Mezi základní charakteristiky portfolia patří jeho \textit{výnosnost}, která je dána jako $r_p=\boldsymbol{X}^\mathrm{T}\boldsymbol{r}=\sum_{j=1}^nX_jr_j$ a~je rovněž náhodnou veličinou.   
Jedním z faktorů pro výběr portfolia v Markowitzově teorii je \textit{očekávaná výnosnost portfolia}, kterou označíme $\mathsf{E}(r_p)=\mu_p$ a je zřejmé, že platí $\mu_p=\sum_{j=1}^nX_j\mu_j$.

\subsection{Riziko}
\textit{Riziko} popisuje míru nejistoty, že se skutečná výnosnost investice bude lišit od očekávané výnosnosti.  
\textit{Riziko aktiva} definujeme jako směrodatnou odchylku výnosnosti aktiva a~označíme ho $\sqrt{\mathsf{D}(r_j)}=\sigma_j$.
Analogicky je \textit{riziko portfolia} definováno jako směrodatná odchylka výnosnosti portfolia a~značeno $\sqrt{\mathsf{D}(r_p)}=\sigma_p$.  

Riziko celého portfolia v~sobě zahrnuje nejen rizika jednotlivých aktiv v~portfoliu, ale také riziko z~vzájemné závislosti výnosností jednotlivých aktiv.
Míra vzájemné závislosti dvou výnosností je mimo jiné popsána jejich kovariancí. 
Kovarianci výnosnosti aktiva~$j$ a~výnosnosti aktiva~$k$ označíme $\mathsf{C}(r_j,r_k)=\sigma_{jk}$.
Snadno se dá dokázat, že $\sigma_p=\sqrt{\sum_{j=1}^n\sum_{k=1}^nX_jX_k\sigma_{jk}}$.
Riziko portfolia je dalším faktorem pro výběr portfolia v~Markowitzově teorii.  


%%%%%%%%%%%%%%%%%%%%%%%%{Klasická teorie portfolia}


\section{Klasická teorie portfolia}\label{KTP}

Základy klasické teorie portfolia položil Harry Markowitz svým článkem v roce 1952 \cite{markowitz}, ve kterém upozornil na nutnost zohledňovat při výběru portfolia nejenom očekávanou výnosnost, ale i riziko změny výnosnosti.
Jeho největším přínosem bylo kvantifikování očekávaného výnosu a rizika portfolia. 
Dalším přínosem byl matematický důkaz výhod diverzifikace, které byly do té doby chápány pouze intuitivně.
Markowitz zkonstruoval množinu všech dostupných portfolií v prostoru výnos-riziko, zavedl pojem \textit{efektivní množina portfolií}, ze které investor vybírá optimální portfolio pomocí indiferenčních křivek popisujících investorův vztah k riziku.

V roce 1958 rozšířil James Tobin \cite{tobin} Markowitzův model o možnost investování do bezrizikového aktiva.
\textit{Bezrizikové aktivum} má jistý výnos a tedy nulové riziko.
To má velký vliv především na efektivní množinu portfolií, která je tvořena tečnou k původní Markowitzově efektivní množině procházející bezrizikovým aktivem.
V bodě dotyku pak leží tzv. tangenciální portfolio.
Postup investora při hledání optimálního portfolia shrnuje Tobinův separační teorém, který se později vžil v souvislosti s CAPM, který bude popsán v další části.% \ref{kapitola_CAPM}.
Investor prvně určí tangenciální portfolio a následně ho zkombinuje s bezrizikový aktivem podle svých rizikových preferencí.

%\subsection{Kapitálový model aktiv - CAPM}\label{kapitola_CAPM}
\textit{Capital Asset Pricing Model} (CAPM) vyvinuli nezávisle na sobě autoři Sharpe \cite{sharpe1964}, Lintner \cite{lintner1965} a Mossin \cite{mossin1966} v letech 1964-1966. CAPM zkoumá chování trhu v případě, že se všichni investoři chovají podle Markowitzovy teorie.
Zároveň vychází z Tobinova modelu, protože zahrnuje bezrizikové aktivum.
CAPM stojí na několika výchozích předpokladech, které se dají shrnout do následujících třech:
\begin{itemize}
\item[-] kapitálový trh je efektivní,
\item[-] investoři při sestavování portfolia využívají Markowitzův přístup,
\item[-] investoři mají homogenní očekávání.
\end{itemize}
Předpoklad homogenity očekávání investorů má za následek, že všichni investoři drží stejné rizikové portfolio.
Optimální portfolia investorů se liší pouze v poměru, v jakém kombinují rizikové portfolio a bezrizikové aktivum v závislosti na svých rizikových preferencích.
Toto shrnuje následující věta.

\begin{veta}(Separační teorém)
Optimální kombinace rizikových cenných papírů může být stanovena bez jakékoliv znalosti investovaných postojů k riziku a výnosnosti.
\end{veta}

Předpokládáme-li rovnováhu trhu, bývá tangenciální portfolio nazýváno jako \textit{tržní portfolio}.
Důležitým důsledkem pak je, že váha každého cenného papíru v tržním portfoliu je rovna jeho tržní ceně. \label{vahy_trznihodnota}


%%%%%%%%%%%%%%%%%%%%%%%%{Dynamická teorie portfolia}

\section{Dynamická teorie portfolia}
Hlavní výhodou dynamického modelu pro výběr aktiv v portfoliu je při\-způsobování se měnícím se podmínkám trhu a proto je více realistický než model statický.
Naproti tomu je dynamický model komplikovanější, hůře interpretovatelný a při jeho využití se nevyhneme složitější matematické teorii, kterou je stochastický kalkul.

\subsection{Matematický úvod}
Na tomto místě připomeneme některé základní pojmy ze stochastické analýzy a značení obvykle užívané v dynamické teorii portfolia. 

\begin{definice}
Stochastický proces $\{W(t):t\ge0\}$ definovaný na pravdě\-podob\-nostním prostoru $(\Omega,\mathcal{A},\mathsf{Pr})$ se nazývá \textit{standardní (jednorozměrný) Wienerův proces} právě tehdy, když jsou splněny následující podmínky
\begin{enumerate}
\item[1.]$W(0)=0$, 
\item[2.]funkce $t\to W(t)$ (trajektorie Wienerova procesu) je spojitá s pravdě\-podob\-ností jedna,
\item[3.]proces $\{W(t):t\ge0\}$ má nezávislé přírůstky, 
\item[4.]pro všechna $t\ge s\ge0$ jsou přírůstky $W(t)-W(s)$ normálně rozdělené se střední hodnotou nula a rozptylem $t-s$.
\end{enumerate}
Standardní \textit{$n$-rozměrný Wienerův proces} je vektorový stochastický proces
$$\boldsymbol{W}(t) = (W_1(t), \dots, W_n(t))^\mathrm{T}$$
jehož složky $W_k(t)$ jsou nezávislé, standardní jednorozměrné Wienerovy procesy.
\end{definice}
Wienerův proces je stavebním kamenem matematického modelování ve finanční matematice. 
Tento stochastický proces se používá pro popis chování ceny aktiva v čase. 

Budeme předpokládat, že proces popisující vývoj ceny aktiva v čase se řídí následujícím modelem
$$\mathrm{d}P(t)=P(t)\mu(t)\mathrm{d}t+P(t)\sigma(t)\mathrm{d}W(t),$$
kde $P(t)$ je cena aktiva v čase $t$, $\mu(t)$ a $\sigma(t)$ jsou stochastické procesy a $W(t)$  je jednorozměrný Wienerův proces.
Tento model předpokládá, že na trhu je pouze jedno rizikové aktivum.

Model pro vývoj ceny aktiv na trhu, který předpokládá existenci $n$ rizikových aktiv je dán jako
$$\mathrm{d}P_j(t)=P_j(t)\mu_j(t)\mathrm{d}t+P_j(t)\sum_{k=1}^{n}\xi_{jk}(t)\mathrm{d}W_k(t),$$
kde $P_j(t)$ je cena aktiva $j$ v čase $t$, $\mu_j(t)$ a $\xi_{jk}(t)$ jsou stochastické procesy a $\boldsymbol{W}(t)=(W_1(t),\dots,W_n(t))^\mathrm{T}$ je $n$-rozměrný Wienerův proces. 
Pro více informací o mnohorozměrných modelech viz \cite{etheridge2002course}.

Chování ceny bezrizikového aktiva je popsáno modelem
$$\mathrm{d}B(t)=B(t)r_f(t)\mathrm{d}t,$$
kde $B(t)$ je cena bezrizikového aktiva v čase $t$ a $r_f(t)$ je bezriziková úroková míra.


\subsection{Mertonův model}
První článek zabývající se dynamickou teorií portfolia napsal Robert Carhart Merton v roce 1971 \cite{merton1971}.
Merton předpokládal platnost rovnovážného modelu CAPM a dál tuto teorii rozšířil o spojitý model pro cenu aktiva využívající stochastické procesy.
Přitom ukázal, že zůstává zachována platnost všech závěrů z teorie CAPM, zejména pak separačního teorému.
Což znamená, že bez újmy na obecnosti můžeme uvažovat jen dvě aktiva -- bezrizikové aktivum s mírou výnosnosti $r_f$ a tržní portfolio (které můžeme považovat za jedno rizikové aktivum). 

Merton předpokládá, že hodnota tržního portfolia $P(t)$ se řídí stochastickým procesem
$$\mathrm{d}P(t)=P(t)\mu_p\mathrm{d}t+P(t)\sigma_p\mathrm{d}W(t)$$
kde $W(t)$ je jednorozměrný Wienerův proces a $\mu_p$, $\sigma_p$ jsou konstanty. 

V každém čase $t$ vybírá investor své optimální portfolio volbou vah portfolia.
Majetek, který investor investuje do optimálního portfolia v čase $t$, označíme $w(t)$.
Váhu tržního portfolia v optimálním portfoliu označíme $X_p(t)$ a váhu bezrizikového aktiva označíme $X_f(t)=(1-X_p(t))$. 
Proces popisující vývoj investorova majetku je dán jako
$$\frac{\mathrm{d}w(t)}{w(t)}=X_p(t)\frac{\mathrm{d}P(t)}{P(t)}+\big(1-X_p(t)\big)r_f\mathrm{d}t.$$
Merton odvodil explicitní řešení této stochastické diferenciální rovnice za předpokladu, že charakteristiky výnosnosti aktiva jsou konstantní.

Avšak Mertonův spojitý rovnovážný model vykazuje inkonzistenci, což dokázali Ohlson a Rosenberg \cite{ohlson} hned v roce 1976.
Ještě před zveřejněním jejich článku byla uveřejněna ve stejném periodiku Mertonova reakce \cite{merton1975}, ve které hájí své poznatky a poukazuje na jejich nepochopení ze strany oponentů.  
Ohlson a Rosenberg \cite {ohlson} poukazují na rozpor mezi předpokladem, že střední hodnota a rozptyl výnosnosti aktiva jsou konstantní, a předpokladem tržní rovnováhy.
Podrobněji bude tento paradox rozebrán v části \ref{paradox}.


%%%%%%%%%%%%%%%%%%%%%%%%%%{Ohlson-Rosenbergův paradox}
\section{Ohlson-Rosenbergův paradox}\label{paradox}
V této kapitole představíme základní poznatky z článku Ohlsona a Rosenberga \cite{ohlson}, který byl publikován v roce 1976 v \textit{Journal of Financial and Quantitative Analysis}.
Ohlson a Rosenberg jako první poukázali na nekonzistenci Mertonova spojitého modelu \cite{merton1971} vznikající kvůli rozporu předpokladů.

\subsubsection{Značení a předpoklady}
Na začátek zavedeme značení a  uvedeme základní definice a předpoklady. 
Uvažujme množinu $n$ různých aktiv na daném trhu a označme $T$ množinu všech uvažovaných časů (diskrétní nebo spojitou).
Nechť $I_t$ značí množinu všech investorů na trhu v čase $t\in T$.
Cenu aktiva $j$ v čase $t$ označíme $P_j(t)$ a $\boldsymbol{P}(t)=(P_1(t),\dots,P_n(t))^\mathrm{T}$ je vektor cen aktiv v čase $t$.
Analogicky označíme $\boldsymbol{N}(t)=(N_1(t),\dots,N_n(t))^\mathrm{T}$ vektor podílů aktiv v čase $t$, potom tržní hodnota aktiva $j$ je rovna $V_j(t)=P_j(t)N_j(t)$.

Míra výnosnosti aktiva $j$, které drží investor v portfoliu v časovém intervalu $(t,t+\Delta t)$ označíme $r_j(t,t+\Delta t)$.
Jak bylo zmíněno výše výnosnost chápeme jako náhodnou veličinu a o její distribuční funkci v čase $t$ vyslovíme tři následující předpoklady:
\begin{enumerate}
\item \label{Stacionarita} \textit{Stacionarita:} distribuční funkce výnosnosti  je stejná ve všech časech $t$,                                                                            
\item \label{Martingalvl} \textit{Martingalová vlastnost:} pro každé $t$ je distribuční funkce výnosnosti nezávislá na všech cenách aktiva pozorovaných do času~$t$,
\item \label{Homogenita} \textit{Homogenita:} distribuční funkce výnosnosti je stejná pro všechny investory z množiny $I_t$. 
\end{enumerate} 
Tyto předpoklady jsou konzistentní s předpoklady CAPM.

Označme $w_{j}(i,t)$ optimální objem prostředků investovaných do aktiva $j$ investorem $i$ v čase $t$.
Nechť $\boldsymbol{w}(i,t)=(w_{1}(i,t),\dots,w_{n}(i,t))$ představuje optimální alokaci prostředků investora $i$ držených v portfoliu všech aktiv.
Funkce $w_{j}(i,t)$ tedy zohledňuje (rizikové) preference investora $i$ v čase $t$.
  
Nyní můžeme definovat dynamickou tržní rovnováhu jako stav, kdy  na\-bíd\-ka a poptávka jsou v dokonalé rovnováze v každém čase $t$.
Tento stav se také nazývá \textit{vyčištění trhu}.
         
\begin{definice}[Dynamická tržní rovnováha]
Řekneme, že kapitálový trh je v \textit{dynamické rovnováze} právě tehdy, když pro každý čas $t\in T$, každé podkladové aktivum $j$ a~každého investora $i$ existuje vektor cen aktiv $\boldsymbol{P}(t)$ takový že
$$\sum_{i\in I} w_{j}(i,t)=N_j(t)P_j(t)=V_j(t).$$
Tyto ceny budeme nazývat \textit{rovnovážné ceny}.
\end{definice}

Důsledek Tobinova separačního teorému shrňme jako vlastnost pro pro\-střed\-ky investované do portfolia, kterou budeme dále využívat v~důkazu inkonzistence předpokladů spojitého modelu pro ceny aktiv.
\begin{definice}\label{vlastnost_separace}[Vlastnost separace]
Vektorová funkce pro optimální alokaci investorových prostředků do všech aktiv na trhu $\boldsymbol{w}(i,t)$ respektuje \textit{vlastnost separace} právě tehdy, když existuje vektor $\boldsymbol{X}=(X_1,\dots,X_n)^\mathrm{T}$ takový, že $\sum_{j=1}^nX_j=1$, a existují skalární parametry $\lambda(i,t)$ takové, že
$$\boldsymbol{w}(i,t)=\lambda(i,t)\boldsymbol{X},$$
pro všechny $i\in I$ a všechny $t\in T$.
\end{definice}
Tedy vektor $\boldsymbol{X}$ představuje váhy tržního portfolia (které se v čase nemění) a parametr $\lambda(i,t)$ zohledňuje rizikové preference investora.

\subsubsection{Rozpor mezi předpoklady}
Budeme-li předpokládat rovnováhu na trhu a stacionaritu distribuční funkce výnosností, dospějeme ke sporu, jak ukazuje následující věta a důsledek.

\begin{veta} \label{T1}
Nechť $\boldsymbol{P}(t)$ pro všechny $t\in T$  jsou rovnovážné ceny.
Dále před\-pokládejme, že na trhu platí vlastnost separace (definice \ref{vlastnost_separace}).
Pak
\begin{equation}\label{T1_eq}
\mathsf{Pr}\left(\frac{N_j(s)P_j(s)}{N_j(t)P_j(t)}=\frac{N_k(s)P_k(s)}{N_k(t)P_k(t)}\right)=1
\end{equation}
pro všechny časy $s,t\in T$  a pro každé aktivum $j$ a $k$.
\end{veta}

\begin{proof}
Podle definice \ref{vlastnost_separace} platí
\begin{equation}\label{CSW}
w_j(i,t)=\lambda(i,t)X_j
\end{equation}
pro všechny časy $t\in T$, všechny investoryl $i\in I$ a pro každé aktivum $j$.

Předpoklad dynamické rovnováhy na trhu dává
\begin{equation}\label{E}
\sum_{i\in I} w_{j}(i,t)=N_j(t)P_j(t)
\end{equation}
pro všechny časy $t\in T$ a pro každé aktivum $j$.

Dosazením vztahu (\ref{CSW}) do rovnice (\ref{E}) a vzhledem k tomu, že váhy $X_j$ jsou stejné pro všechny investory $i\in I$, dostáváme
$$N_j(t)P_j(t)=\sum_{i\in I}\lambda(i,t)X_j=X_j\sum_{i\in I}\lambda(i,t).$$
Proto platí
$$\frac{N_j(t)P_j(t)}{N_k(t)P_k(t)}=\frac{X_j\sum_{i\in I}\lambda(i,t)}{X_k\sum_{i\in I}\lambda(i,t)}=\frac{X_j}{X_k},$$
což je nezávislé na čase $t$.         
Z tohoto tvrzení zřejmě plyne vztah (\ref{T1_eq}).
\end{proof}

\begin{dusledek}\label{OR_dusledek}
Nechť 
$$r_j(t,t+\Delta t)=\frac{P_j(t+\Delta t)-P_j(t)}{P_j(t)}=\frac{P_j(t+\Delta t)}{P_j(t)}-1.$$
Budeme-li předpokládat $N_j(t)=N_j$ pro každé aktivum $i$ a všechna $t\in T$. 
Z~věty \ref{T1} plyne
\begin{align*}
\mathsf{Pr}\left(\frac{P_j(s)}{P_j(t)}=\frac{P_k(s)}{P_k(t)}\right)=1&\Longrightarrow\mathsf{Pr}\left(\frac{P_j(t+\Delta t)}{P_j(t)}=\frac{P_k(t+\Delta t)}{P_k(t)}\right)=1 \\
&\Longrightarrow\mathsf{Pr}\left(r_j(t,t+\Delta t)=r_k(t,t+\Delta t)\right)=1,
\end{align*}
pro všechna $t\in T$ a pro všechna aktiva $j$ a $k$.
Proto jsou aktiva $j$ a $k$ vzájemně dokonale zastupitelné.
\end{dusledek}
Důsledek \ref{OR_dusledek} vede k degeneraci trhu, což poukazuje na významný rozpor mezi předpoklady.

Ve své době se Ohlson a Rosenberg nesetkali s pochopením.
Důkazem toho je okamžité odmítnutí jejich závěrů ze strany Mertona publikované v článků \cite{merton1975}, kde tvrdí, že rozporu bylo docíleno jen díky nereálným předpokladům. 
Ve skutečnosti reaguje jen na důsledky nikoli na důkaz stěžejní věty článku \cite{ohlson}.  
Za povšimnutí stojí, že Mertonova reakce vychází ještě před samotným článkem Ohlsona a Rosenberga.

%%%%%%%%%%%%%%%%%%%%%%%%%%{Stochastická teorie portfolia}

\subsection{Stochastická teorie portfolia}
V této podkapitole jsou shrnuty některé základní poznatky Stochastické teorie portfolia, která byla inspirací pro studium spojitých procesů v teorii portfolia uvedených v této práci.
Budeme diskutovat o výhodách a nevýhodách tohoto přístupu.

Pojem \textit{Stochastická teorie portfolia} (SPT) zavedl Robert Fernholz ve své monografii \cite{fern}.
Jedná se o matematickou teorii, která zkoumá chování portfolia jakožto i strukturu kapitálového trhu.
Hlavní výhodou oproti klasické teorii portfolia jsou méně  striktní předpoklady, které dovolují konzistentní přístup bez výskytu Ohlson-Rosenbergova paradoxu \cite{ohlson}.

Na rozdíl od Mertonova modelu se v SPT nepředpokládá tržní rovnováha.
Parametry stochastického procesu, kterým se řídí cena aktiva, jsou v SPT také stochastické procesy nikoli konstanty jako v Mertonově přístupu.
Dokonce, na rozdíl od klasického přístupu, se v SPT neklade důraz na předpoklad neexistence tržní arbitráže, přičemž jsou studovány vlastnosti trhu vedoucí k existenci arbitráže. 

Pro vývoj ceny aktiva Fernholz používá následující spojitý logaritmický model
$$\mathrm{d}\log P_j(t)=\gamma_j(t)\mathrm{d}t+\sum_{k=1}^{n}\xi_{jk}(t)\mathrm{d}W_k(t),$$
kde $P_j(t)$ je cena aktiva $j$ v čase $t$, $\gamma_j(t)$ a $\xi_{jk}(t)$ jsou stochastické procesy a $\boldsymbol{W}(t)=(W_1(t),\dots,W_n(t))^\mathrm{T}$ je $n$-dimenzionální Wienerův proces. 
Proces $\gamma_j(t)$ se nazývá \textit{míra růstu} a $\xi_{jk}(t)$ se nazývá proces \textit{volatility}.
Míru růstu v logaritmickém modelu je možno odvodit z míry výnosnosti ve standardním modelu a naopak podle následujícího vztahu
$$\mu_j(t)=\gamma_j(t)+\frac12\sum_{k=1}^{n}{\xi_{jk}}^2(t).$$
Robert Fernholz uvádí, že v dlouhém časovém horizontu je chování hodnoty portfolia lépe popsáno mírou růstu než mírou výnosnosti.
Uvažujeme-li kovarianční matici $\boldsymbol{\Sigma}$ výnosností aktiv, pak matice volatilit $\boldsymbol{\xi}$ je definována jako
$$\boldsymbol{\Sigma}=\boldsymbol{\xi}\boldsymbol{\xi}^\mathrm{T}.$$
Tedy $\boldsymbol{\xi}$ je maticová odmocnina z $\boldsymbol{\Sigma}$.

Jedním ze základních pojmů v SPT jsou \textit{portfolio generující funkce}, pomocí nichž jsou tvořeny portfolia, které mají dobře definované výnosnosti. Podrobnější informace je možno najít v knize \cite{fern}.



%%%%%%%%%%%%%%%%%%%%%%%%%%%%%%%

\section{Paradox v teorii portfolia a Věta o~separaci do~dvou fondů} \label{john}
Nekonzistencí obvyklých předpokladů teorie portfolia se v roce 2009 zabýval také John Stalker profesor na univerzitě v Princetonu. 
V preprintu \cite{john} uvádí algebraický důkaz inkonzistence předpokladů, kde využívá Tobinovu \textit{Větu o~separaci do~dvou fondů}. 
V následujícím textu budou tyto myšlenky rozpracovány.

Uvažujme následující předpoklady obvyklé pro teorii portfolia:
\begin{enumerate}
\item \label{predpoklad_konstantnosti_vynosu_a_rizika} Ceny podkladových aktiv se řídí It\^oovými procesy, které mají konstantní očekávané míry výnosnosti a konstantní kovariance výnosností. 
\item Neexistují žádná omezení na množství aktiv držených investorem. Aktiva jsou navíc nekonečně dělitelná.
\item Neexistují žádné transakční náklady a cena prodeje a nákupu podkladového aktiva se neliší.
\item Investoři se chovají podle Markowitzovy teorie, což znamená, že minimalizují riziko svého portfolia pro dané očekávané míry výnosnosti.
\item \label{predpoklad_hodnost} Na trhu existuje pouze jedno bezrizikové aktivum. Kovarianční matice rizikových aktiv má maximální hodnost.
\item \label{predpoklad_konstantnosti_N} Na trhu existuje rovnováha a každý investor nabízející aktivum najde kupce a naopak.
\end{enumerate}

Dále dokážeme, že tyto předpoklady jsou nekonzistentní pro trh s více než dvěma rizikovými aktivy.
K tomu využijeme Větu o~separaci do~dvou fondů, kterou poprvé dokázal  James Tobin ve svém článku \cite{tobin}. 
Tobin předpokládal platnost předpokladů CAPM a zároveň zahrnul existenci bezrizikového aktiva.
Naproti tomu v knize Williama Sharpeho \cite{sharpe} a v článku Roberta Mertona \cite{merton} najdeme podobné závěry bez předpokladu existence bezrizikového aktiva.
Merton navíc tyto myšlenky zobecňuje pro více než dva fondy \cite{merton1973}.
Podrobnějším rozbor teorie o dvou fondech je uveden v knize \cite{cass1970structure}.
\begin{veta}[Věta o~separaci do~dvou fondů]
Efektivní portfolio každého investora lze získat jako kombinaci dvou fondů.
\end{veta}

Na trhu s $m$ podkladovými aktivy uvažujme dvě portfolia (neboli dva fondy) a jejich váhy označme $\boldsymbol{a}=(a_1,\dots,a_m)^\mathrm{T}$ a $\boldsymbol{b}=(b_1,\dots,b_m)^\mathrm{T}$. Z těchto dvou fondů pak můžeme sestavit optimální portfolia všech investorů na trhu.
Množinu všech investorů označme $I$ a pro každého investora uvažujme jeho rizikové preference, které určují jeho optimální portfolio, a popišme je pomocí parametrů $\alpha_i(t)$ a $\beta_i(t)$.
Optimální portfolio $i$-tého investora bude zahrnovat $\alpha_ia_j+\beta_ib_j$ množství $j$-té akcie.

Váhy $\boldsymbol{a}$ a  $\boldsymbol{b}$ dvou uvažovaných fondů závisí pouze na očekávaných vý\-nos\-nostech a riziku podkladových aktiv na trhu, které jsou v čase konstantní dle předpokladu \ref{predpoklad_konstantnosti_vynosu_a_rizika}.
Parametry $\alpha_i(t)$ a $\beta_i(t)$ závisí na individuálních preferencích jednotlivých investorů (např. rizikové preference nebo objem investic) a můžou se lišit. Tyto preference se mohou v čase měnit a tak parametry mohou záviset i na čase.

Označme $A(t)$ součet všech parametrů $\alpha_i(t)$ pro všechny investory a analogicky $B(t)$ součet parametrů $\beta_i(t)$ .
Tedy $A(t)=\sum_{i\in I}\alpha_i(t)$ a $B(t)=\sum_{i\in I}\beta_i(t)$.
Pak celková hodnota držená všemi investory v $j$-tém aktivu je $Aa_j+Bb_j$ a musí být rovna tržní hodnotě aktiva.
$$A(t)a_j+B(t)b_j=P_j(t)N_j,$$
kde $P_j(t)$ je cena podkladového aktiva na trhu v čase $t$ a $N_j$ je jeho dostupné množství.
Přičemž $A(t),B(t)$ a $ P_j(t)$ jsou funkce času a $N_j$ je konstantní v čase podle předpokladu \ref{predpoklad_konstantnosti_N}.
Pro změnu ceny aktiva $j$ mezi časem $t$ a časem $t+\Delta t$ platí
\begin{displaymath}
\Delta P_j = \frac{a_j}{N_j}\Delta A + \frac{b_j}{N_j}\Delta B,
\end{displaymath}
kde $\Delta A = A(t+\Delta t) - A(t)$ and $\Delta B = B(t+\Delta t)- B(t)$.

Uvažujme míru výnosnosti podkladového aktiva $j$ za čas $\Delta t$
\begin{displaymath}
r_j(t,t+\Delta t)=\frac{\Delta P_j}{P_j(t)}=\frac{a_j}{N_jP_j(t)}\Delta A + \frac{b_j}{N_jP_j(t)}\Delta B
\end{displaymath}
a kovarianční matici výnosností podkladových aktiv držených v portfoliu
$$\boldsymbol{\Sigma}(\boldsymbol{r})=
\begin{pmatrix}
\mathsf{D}(r_1)  & \ldots & \mathsf{C}(r_1,r_m)  \cr \vdots & \ddots & \vdots\cr \mathsf{C}(r_m,r_1)   & \ldots & \mathsf{D}(r_m) 
\end{pmatrix}.$$
Vzhledem k vlastnostem kovariance (viz \cite{andel}) můžeme kovarianční matici výnosností rozložit na součin 
$$
%\begin{displaymath}
\boldsymbol{\Sigma}(\boldsymbol{r})= 
\begin{pmatrix}
\frac{a_1}{N_1P_1} & \frac{b_1}{N_1P_1}\cr \vdots & \vdots\cr \frac{a_m}{N_mP_m} & \frac{b_m}{N_mP_m}
\end{pmatrix}
%\end{displaymath}
\times
%\begin{displaymath*}
\begin{pmatrix}
\frac{\mathsf{C}(\Delta A, \Delta A)}{\Delta t} &
\frac{\mathsf{C}(\Delta A, \Delta B)}{\Delta t} \cr
\frac{\mathsf{C}(\Delta B, \Delta A)}{\Delta t} & \frac{\mathsf{C}(\Delta B,
\Delta B)}{\Delta t} 
\end{pmatrix}
%\end{displaymath*}
\times
%\begin{displaymath*}
\begin{pmatrix}
\frac{a_1}{N_1P_1} & \ldots & \frac{a_m}{N_mP_m}\cr \cr \frac{b_1}{N_1P_1} & \ldots &
\frac{b_m}{N_mP_m}
\end{pmatrix}.
%\end{displaymath*}
$$

Dále platí, že hodnost součinu matic je menší nebo rovna maximální z hodností matic jeho činitelů.
Vzhledem k tomu nemůže mít kovarianční matice výnosností podkladových aktiv držených v portfoliu hodnost větší než dvě.
Avšak to je v rozporu z předpokladem \ref{predpoklad_hodnost}, pokud uvažujeme portfolio s více než dvěma podkladovými aktivy.

%%%%%%%%%%%%%%%%%%%%%%%%%%%%%%%

\section{Spojitý rovnovážný model s očekávanou výnos\-ností závislou na ceně aktiva }
V kapitolách \ref{paradox} a \ref{john} jsme ukázali, že v teorii portfolia běžně používaný dynamický model není konzistentní.
Proto se chceme pokusit navrhnout takový model, který by nebyl touto inkonzistencí předpokladů zatížen.
Uvažujme tedy spojitý model, který bude mít méně omezující předpoklady než Mertonův modelu.
Model, který předkládáme v této kapitole, předpokládá rovnováhu na trhu, nicméně očekávaná míra výnosnosti není konstantní v čase.

Na základě poznatků z teorie portfolia budeme hledat vztah mezi oče\-ká\-vanou výnosností a cenou aktiva, který bychom mohli použít v modelu.
Nechť $\boldsymbol{P}$ je vektor cen $n$ podkladových aktiv.
Předpokládejme, že tyto ceny se řídí následující stochastickou diferenciální rovnicí (SDR):
\begin{equation} \label{SDE}
\mathrm{d}P_j(t)=P_j(t)\mu_j(t)\mathrm{d}t+P_j(t)\sum_{k=1}^{n}\xi_{jk}(t)\mathrm{d}W_k(t),
\end{equation}
kde $\boldsymbol{W}(t)=(W_1(t),\dots,W_n(t))^\mathrm{T}$ je $n$-rozměrný Wienerův proces, $\mu_j(t)$ je očekávaná míra výnosnosti podkladového aktiva $j$ a $\xi_{jk}(t)$ jsou volatility podkladových aktiv. 
Pro zjednodušení předpokládejme, že výnosnost bezrizikového aktiva $r_f=0$.
Nechť $\boldsymbol{X}=(X_1,\dots,X_n)^\mathrm{T}$ je vektor vah v tržním portfoliu, který je dán jako
\begin{equation} \label{vahy}
\boldsymbol{X}=\frac{\boldsymbol{\Sigma}^{-1}\boldsymbol{\mu}}{\boldsymbol{1}^\mathrm{T}\boldsymbol{\Sigma}^{-1}\boldsymbol{\mu}},
\end{equation}
kde $\boldsymbol{1}$ je vektor jedniček, $\boldsymbol{\mu}=(\mu_1,\dots,\mu_n)^\mathrm{T}$ je vektor očekávaných výnosností podkladových aktiv a $\boldsymbol{\Sigma}$ je kovarianční matice výnosností podkladových aktiv (viz \cite{fabozzi}).

Jak jsme ukázali v kapitole \ref{vahy_trznihodnota}, váhy tržního portfolia se rovnají relativním tržním hodnotám podkladových aktiv v tržním portfoliu, tedy
\begin{equation} \label{market_value}
\boldsymbol{X}=\frac{\boldsymbol{V}}{\boldsymbol{1}^\mathrm{T}\boldsymbol{V}},
\end{equation}
kde $\boldsymbol{V}$ je vektor tržních hodnot podkladových aktiv na trhu.

Ze vztahů (\ref{vahy}) a (\ref{market_value}) dostáváme $\boldsymbol{V}=\boldsymbol{\Sigma}^{-1}\boldsymbol{\mu}$ a proto očekávané výnos\-nosti vyjádříme jako
\begin{equation} \label{mu}
\boldsymbol{\mu}=\boldsymbol{\Sigma}\boldsymbol{V}.
\end{equation}

Předpokládejme nyní, že uvažovaná podkladová aktiva jsou nekorelovaná, tedy matice $\boldsymbol{\Sigma}$ je diagonální.
Příkladem takových aktiv by mohly být akcie podniků z velmi odlišných odvětví.
Další zjednodušující předpoklady, které pro tuto chvíli zavedeme, bude neměnnost rizika aktiv $\sigma_j$ v průběhu času a také hodnoty aktiv $N_j$ nechť jsou v čase konstantní.
Připomeňme, že tržní hodnota  $V_j(t)=P_j(t)N_j(t)$ a $\sigma_{j}=\xi_{jj}$.

Dosadíme-li vztah (\ref{mu}) do stochastické diferenciální rovnice (\ref{SDE}), pak s ohledem na předpoklady učiněné výše dostáváme
$$ 
\mathrm{d}P_j(t)=P_j^2(t){\sigma_{j}}^2N_j\mathrm{d}t+P_j(t)\sigma_{j}\mathrm{d}W_j(t).
$$
Dále budeme hledat vztah pro očekávané hodnoty cen podkladových aktiv.
Ze SDR (\ref{SDE}) plyne
\begin{equation}
\mathsf{E}\big(\mathrm{d}P_j(t)\big)=\mathsf{E}\big(P_j^2(t){\sigma_{j}}^2N_j\mathrm{d}t+P_j(t)\sigma_{j}\mathrm{d}W_j(t)\big)=\mathsf{E}\big(P_j^2(t){\sigma_{j}}^2N_j\mathrm{d}t\big),
\end{equation}
protože z vlastností Wienerova procesu máme $\mathsf{E}\big(\mathrm{d}W_j(t)\big)=0$.
Čímž se stochastická diferenciální rovnice zjednodušuje na obyčejnou diferenciální rovnici (ODR), která je řešitelná pomocí separace proměnných.
\begin{align} 
{\mathrm{d}\mathsf{E}\big(P_j(t)\big)}&={\sigma_{j}}^2N_j\mathsf{E}\big(P_j^2(t)\big)\mathrm{d}t \notag\\
&={\sigma_{j}}^2N_j\big[\mathsf{D}\big(P_j(t)\big)+\mathsf{E}^2\big(P_j(t)\big)\big]\mathrm{d}t \notag\\
&={\sigma_{j}}^2N_j\big[{\sigma_{j}}^2+\mathsf{E}^2\big(P_j(t)\big)\big]\mathrm{d}t \label{ODE}
\end{align}  
Řešení ODR ($\ref{ODE}$) je
\begin{equation} 
\mathsf{E}\big(P_j(t)\big)=\sigma_j\mathrm{tan}\big({\sigma_j}^3N_jt\big).
\end{equation}
Docházíme tedy k závěru, že řešení má singularitu v konečném čase.
Jelikož pro ${\big(\sigma_j}^3N_jt\big)\to\frac\pi2$ platí $\mathsf{E}\big(P_j(t)\big)\to\infty$. 
To by mohlo být interpretováno jako chování cenových bublin na trhu.
K těm dochází v případě, že investoři ve velkém měřítku investují do aktiv s nadsazenými cenami, což vede k dalšímu umělému zvyšování cen. 
Takový stav nevyhnutelně vede k nečekanému krachu ceny (tzv. prasknutí bubliny).
Problematické je takovou bublinu na trhu poznat, jelikož jsme schopni ji identifikovat až zpětně při extrémním poklesu cen, který je důsledkem mohutného odprodeje aktiv.

V další části představíme model s méně přísnými předpoklady.
Připusťme nyní možnost závislosti mezi jednotlivými aktivy, tedy nechť matice $\boldsymbol{\Sigma}$ není diagonální.
Za těchto předpokladů dosaďme vztah (\ref{mu}) do SDR \eqref{SDE}, čímž obdržíme
\begin{equation*} 
 \mathrm{d}P_j(t)=P_j(t)\sum_{k=1}^{n}P_k(t)N_k(t)\sigma_{jk}\mathrm{d}t+P_j(t)\sum_{k=1}^{n}\xi_{jk}(t)\mathrm{d}W_k(t).
\end{equation*}
Budeme-li opět uvažovat očekávanou hodnotu  cen podkladových aktiv do\-stá\-váme vztah
\begin{equation*} 
\mathsf{E}\big( \mathrm{d}P_j(t)\big)=\mathsf{E}\left(P_j(t)\sum_{k=1}^{n}P_k(t)N_k(t)\sigma_{jk}\mathrm{d}t\right),
\end{equation*}
z něhož dostaneme následující soustavu obyčejných diferenciálních rovnic
\begin{align*}
\mathrm{d}\mathsf{E}\big(P_j(t)\big)&=N_j\left[\sum_{k=1}^{n}{\sigma_{jk}}^2+\mathsf{E}\big(P_j(t)\big)\sum_{k=1}^{n}\sigma_{jk}\mathsf{E}\big(P_k(t)\big)\right]\mathrm{d}t.
\end{align*}  
Avšak tato soustava je nelineární a bohužel neexistuje její analytické řešení.
Neobejdeme se tedy bez použití numerických metod.


%%%%%%%%%%%%%%%%%%{Credit Valuation Adjustment}

\chapter{Credit Valuation Adjustment}

Credit Valuation Adjustment (CVA)

Kreditní prirážka k tržnímu ocenení

%%%%%%%%%%%%%%%%%%


%%%%%%%%%%%%%%%%%%%%%%%%%%%%%%%


\nocite{}  %umisti do literatury i necitovanou polozku a \nocite{*} umisti do literatury vsechny polozky z bibtex databaze 

\addcontentsline{toc}{chapter}{Literatura}
%\bibliographystyle{plain}
\bibliographystyle{abbrv}
\bibliography{bibliography}

\end{document}
